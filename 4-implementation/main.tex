\section{Implementation}

\subsection{Picking a programming language}
We decided to program AiryScript in Scala fairly early on for the following
reasons.
\begin{itemize}
  \item We wanted to be able to really define our own language syntax.
    Generally, this means having our own parser and not merely calling ‘eval’ in
    some other language. Also, all of the members of our team prefer typed
    languages. Therefore, we also wanted to create a typed language. We could
    not see how to do this without making our own parser. We were already in
    possession of some basic parser code for Scala.
  \item Some of the members of our team were already familiar with Scala.
  \item All of the members of our team were already familiar with Java, which
    makes it fairly easy to learn (the very basics of) Scala.
  \item Scala has very good support for pattern matching through case classes
    (i.e. tagged unions). We make extensive use of this in our code.
  \item Some of the members of our team had been looking for a good opportunity
    to get to know Scala for some time.
\end{itemize}


\subsection{Implementation overview}

We divided the implementation in 3 parts:
\begin{enumerate}
\item The parser: Responsible for parsing a string into the right operation, or
returning a parse error if the given expression was incorrect.
\item The operation handler: The actual implementation of all the different
operators. This part is responsible for verifying all the arguments of an
operation and either trowing an error if these arguments were
incorrect/incomplete or performing the operation. Both for verifying aswell as
performing an operation database acces is required.
\item The database: We used an SQL database to backup the system. The
implementation of this database closely resembles the domain model.
\end{enumerate} 
	
\par
All of the operations and types are resembled by a single case class in scala,
which captures all of the required information to use this. The parser will
parse a string into an operation, after which this is passed to the second
phase. The handler can then handle these operations by pattern matching over the
different kinds of operators and their respective arguments.


\subsection{Scala Parser}

The responsibility of our parser is to turn a string into one of the possible
operations, or throw an error if the syntax is incorrect. We extended the
\class{StandardTokenParsers} class available in the standard scala api, this
allowed to make a clean, extendable parser written in pure functional code.

We mentioned the 

\subsubsection{Tokens} 

The first step in the process of parsing is translating the string into a list
of tokens. Special sequences within the string will be converted into special
tokens aswell.

\par
There are 4 kinds of tokens:
\begin{description} 
\item[keywords] There are 2 sort of sequences that can be parsed into a keyword
token:
	\begin{description}
	\item[delimiters] whenever one of these sequences is recognized it is parsed
	into a delimiter-token. The delimiters in our implementation are: \sn{\{},
	\sn{\}}, \sn{,}, \sn{:}.
	\item[reserved keywords] These sequences are only parsed into their
	corresponding tokens if they are not part of a larger word. Concretely this
	means that they are only parsed if they are surrounded by white space or
	delimiters. The reserved keywords in our implementation are all the words used
	to describe operations: \sn{ADD}, \sn{REMOVE}, \sn{CITY}, \sn{AIRPORT}, etc.
	\end{description}
\item[numeric literal] Whenever a sequence of numbers is spotted, this will be
parsed into a numeric literal token.
\item[string literal] Whenever a sequence between quotation marks is found it is
converted as a whole into a string literal, so within this sequence no further
splitting is done. This is usefull for providing strings that contain spaces
such as \sn{New York}.
\item[identifier] Basically any kind of words that were not described before,
they are sepeterated by delimiters and white space.
\end{description}
	
\par
Parsing the string \sn{ADD CITY \{name:Brussels, short:BRU\}} will result into
the following list: \sn{keyw ADD}, \sn{keyw CITY}, \sn{"keyw \{"}, \sn{ident
name}, \sn{keyw :}, \sn{ident Brussels}, \sn{keyw ,}, \sn{ident short},
\sn{keyw  :}, \sn{keyw BRU}, \sn{keyw \}}.
 
\par
Note that the actual creation of these tokens is completely done by the
\class{StdTokenParsers} class that we extended. The only thing we provide is the
list with delimiters and reserved keywords.


\subsubsection{Parsing operators}

Ones a list of tokens is obtained the actual parsing can start. We can use the
syntax provided by the StdTokenParsers class to do this.

\par
The first step in parsing an operation is quite simple. Since every operator
has a unique syntax to call it we just try to match the current string with
a list of all the possible operators. A fragment of this part:

\begin{lstlisting}
def parseOp(): Parser[Operation]  =
    "ADD" ~> "CITY" ~> parseCityData ^^ {c => AddCity(c)} |
    "ADD" ~> "AIRPORT" ~> parseAirportData ^^ {a => AddAirport(a)}
    //...
\end{lstlisting}

\par
The \class{StanderdTokensParsers} provides us with a range of special operations
that can easily be used to parse tokens. The \sn{\textasciitilde>} in the
fragment above is used to match a keyword token corresponding to that string,
the result obtained form this match is however discarted. The
\sn{\textasciicircum\textasciicircum} is used to retain the result from this
line of match operators. 

\par
The \sn{parseCityData} in the code above is simple a val which contains the
definition to parse a \class{City\_data}. If the provided string actually
matches with this definition then an actual \class{City\_data} object will be
constructed and passed to the function after the \sn{\textasciicircum\textasciicircum}
operator. We use this to construct a new \class{AddCity} case class, which
maps 1-to-1 with the definition of add city as explained in
\ref{sec:operations}.

\par
The following string will match the first operator in the definition above, the
definition in \sn{parseCityData} will create the actual city with name
\sn{"Brussels"} which is stored inside a new \class{AddCity}.

\begin{lstlisting}
"ADD CITY {name:Brussels}"
\end{lstlisting}


\subsubsection{Parsing types}

Parsing types is analoug to parsing operators, only that a few additional
problems have to be tackled here. Let us look at the example of parsing city
data. For this example sn{\{name:Brussels, short:BRU\}} and \sn{\{short:BRU,
name:Brussels\}} are both valid descriptions of city data. The order in which
the attributes appear in the definition do not matter. Also the user does not
have to explicitly mention what kind data he is providing. That means that we
need to check if the data he provided matches any of the possible formats for
this operation. Since data types can be nested we need to check all
possibilities and backtrack when a mismatch is spotted.

\par
Luckily the implementation for backtracking is already present in the parser we
extended. We have to use the right operators however to trigger this. Whenever
we use the \sn{\textasciicircum?} operator it will behave the same as a the
\sn{\textasciicircum\textasciicircum} operator except that when the result cannot
match we will backtrack instead of fail parsing.

\par
We implemented a general way of dealing with the order of attributes. Whenever
we start parsing any kind of data object we pull out all of its attributes and
place them in a map. This map can thereafter be consulted to see if all required
attributes were provided, regardless of the order they were originally provided
in.

\subsection{Operation handler} 
\subsection{SQL Database}
