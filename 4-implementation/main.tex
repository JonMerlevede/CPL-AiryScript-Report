\section{Implementation}

\subsection{Implementation overview}

We divided the implementation in 3 parts:
- The parser:
	Responsible for parsing a string into the right operation,
	or returning a parse error if the given expression was incorrect.
- The operation handler:
	The actual implementation of all the different operators.
	This part is responsible for verifying all the arguments of an operation
	and either trowing an error if these arguments were incorrect/incomplete or
	performing the operation. Both for verifying aswell as performing an operation
	database acces is required.
- The database:
	We used an SQL database to backup the system. The implementation of this
	database closely resembles the domain model.
	
All of the operations and types are resembled 1 to 1 by a case class in scala.
The parser will parse a string into an operation, after which this is passed
to the second phase. The handler can then handle these operations by pattern
matching over the different kinds of operators and their respective arguments.
The 

\subsection{Implementation of a parser in Scala}

The responsibily of our parser is to turn a string into one of the possible
operations, or throw an error if the syntax is incorrect. We extended the
StandardTokenParsers class available in the standard scala api, this allows to
make a clean parser written in pure functional code.

The first step in the process of parsing 



\subsection{Implementation of a parser in Scala}
