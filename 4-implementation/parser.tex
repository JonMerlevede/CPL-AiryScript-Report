\subsection{Scala Parser}

The responsibily of our parser is to turn a string into one of the possible
operations, or throw an error if the syntax is incorrect. We extended the
StandardTokenParsers class available in the standard scala api, this allows to
make a clean, extendable parser written in pure functional code.

\subsubsection{Tokens} 

The first step in the process of parsing is translating the string into a list
of tokens. Special sequences within the string will be converted into special
tokens aswell.

\par
There are 4 kinds of tokens:
\begin{description} 
\item[keywords] There are 2 sort of sequences that can be parsed into a keyword
token:
	\begin{description}
	\item[delimiters] whenever one of these sequences is recognized it is parsed
	into a delimiter-token. The delimiters in our implementation are: \sn{\{},
	\sn{\}}, \sn{,}, \sn{:}.
	\item[reserved keywords] These sequences are only parsed into their
	corresponding tokens if they are not part of a larger word. Concretely this
	means that they are only parsed if they are surrounded by white space or
	delimiters. The reserved keywords in our implementation are all the words used
	to describe operations: \sn{ADD},\sn{CITY}, \sn{AIRPORT}, etc.
	\end{description}
\item[numeric literal] Whenever a sequence of numbers is spotted, this will be
parsed into a numeric literal token.
\item[string literal] Whenever a sequence between quotation marks is found it is
converted as a whole into a string literal, so within this sequence no further
splitting is done. This is usefull for providing strings that contain spaces
such as \sn{New York}.
\item[identifier] Basically any kind of words that were not described before,
they are sepeterated by delimiters and white space.
\end{description}
	
\par
Parsing the string \sn{ADD CITY \{name:Brussels, short:BRU\}} will result into
the following list: \sn{ADD}, \sn{CITY}, \sn{"\{"}, \sn{ident name}, \sn{":"},
\sn{ident Brussels}, \sn{","}, \sn{ident short}, \sn{":"}, \sn{BRU}, \sn{"\}"}.
 
\par
Note that the actual creation of these tokens is completely done by the
StdTokenParsers class that we extended. The only thing we provide is the list
with delimiters and reserved keywords.


\subsubsection{Parsing operators}

Ones a list of tokens is obtained the actual parsing can start. We can use the
syntax provided by the StdTokenParsers class to do this.

\par
The first step in parsing an operation is quite simple. Since every operator
has a unique syntax to call it we just try to match the current string with
a list of all the possible operators. A fragment of this part:

\begin{lstlisting}
def parseOp(): Parser[Operation]  =
    "ADD" ~> "CITY" ~> parseCityData ^^ {c => new AddCity(c)} |
    "ADD" ~> "AIRPORT" ~> parseAirportData ^^ {a => new AddAirport(a)} |
    \ldots
\end{lstlisting}

\par
StanderdTokensParsers provides us with range of special operations that can
easily be used to parse tokens. The \sn{\textasciitilde>} in the fragment above is used to
match a keyword token corresponding to that string, the result obtained form
this match is however discarted. The \sn{\textasciicircum\textasciicircum} is
used to retain the result from this line of match operators. In the first line of our example above this will
simply be the result of parseCityData, since the result of the other match
operations got discarted.


\subsubsection{Parsing types}

Parsing types is analoug to parsing operators, only that a few additional
problems have to be tackled here. Let us look at the example of parsing city
data. For this example sn{\{name:Brussels, short:BRU\}} and \sn{\{short:BRU,
name:Brussels\}} are both valid descriptions of city data. The order in which
the attributes appear in the definition do not matter. Also the user does not
have to explicitly mention what kind data he is providing. That means that we
need to check if the data he provided matches any of the possible formats for
this operation. Since data types can be nested we need to check all
possibilities and backtrack when a mismatch is spotted.

\par
Luckily the implementation for backtracking is already present in the parser we
extended. We have to use the right operators however to trigger this. Whenever
we use the \sn{\textasciicircum?} operator it will behave the same as a the
\sn{\textasciicircum\textasciicircum} operator except that when the result cannot
match we will backtrack instead of fail parsing.

\par
We implemented a general way of dealing with the order of attributes. Whenever
we start parsing any kind of data object we pull out all of its attributes and
place them in a map. This map can thereafter be consulted to see if all required
attributes were provided, regardless of the order they were originally provided
in.
