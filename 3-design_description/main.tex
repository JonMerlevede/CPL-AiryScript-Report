\section{Design description}
This section describes the development process of the actual design of the DSL
implemented by AiryScript.

\subsection{Picking a programming language}

\subsection{Devising a language syntax}

\subsubsection{Iteration one}

\subsubsection{Iteration two}

\subsubsection{Iteration three}


\subsection{Describing and creating types}
\label{sec:syntax_spec}
This section specifies how to create structures in AiryScript and explains the
notation that Section \ref{sec:operations} uses to specify types. Even though
AiryScript does not require its users to declare types or provide them with
the possibility to define their own types, knowing how to describe these types
and how the typing system of AiryScript works is important, since AiryScript is
a typed language.

\paragraph{Primitive types}
AiryScript supports strings and integers as primitive types. The statements
\begin{quote}
  \tye{5}\\
  \tye{502349}
\end{quote}
define the integers 5 and 502349. The statements
\begin{quote}
  \tye{IAmFreeOfSpacesAndIThereforeDoNotRequireQuotation}\\
  \tye{"I am not free of spaces and therefore I do require quotation"}\\
  \tye{"IAmFreeOfSpacesYetICanStillUseQuotation"}
\end{quote}
each define a string. Users should surround strings that contain spaces with
quotation marks.


\paragraph{Structures}
AiryScript also supports structures. Structures consist of one or several
properties. Properties have a name and a value. A name is always a string that
does not contain any spaces. A value is either a primitive value or another
structure. Names are separated from values by a colon, properties are separated
from each other by commas and structures are defined within curly brackets. As
an example,
\begin{quote}
  \ins{a: 10, b: 5}
\end{quote}
defines a structure with a property named \tye{a} that has an integer primitive
value of ten and a property named \tye{b} that has an integer primitive value of
five. AiryScript is not sensitive to whitespace or newlines. For example,
\begin{quote}
  \begin{verbatim}
{  a:10, b       :
                    5}
  \end{verbatim}
\end{quote}
defines the same structure as before.


\paragraph{Basic structure types}
In the documentation,
\begin{quote}
  \ty{A}\tye{ = \ins{a: Int, b: String}}
\end{quote}
defines \ty{A} to be shorthand for the type that collects all structures that
have properties named \tye{a} and \tye{b} with respective types \tye{Int} and
\tye{String}. This simple definition completely describes a type. If AiryScript
would allow users to define their own types and operations, this would be the
syntax it would use.

As an example, the following two lines each define a structure that is an
instance of \ty{A}.
\begin{quote}
  \ins{a: 10, b: tomatoes}\\
  \ins{b: "bottles of beer", a: 10}\\
\end{quote}


\paragraph{Types with optional properties}
In the documentation,
\begin{quote}
  \ty{B}\tye{ = \ins{a: Int, ?b: Int, ?c: String}}
\end{quote}
defines \ty{B} to be shorthand to the type that collects all structures that
have properties named \tye{a} with type \ty{Int}, and that optionally has two
more properties with names \tye{b} and \tye{c} and values with respective types
\ty{Int} and \ty{String}. None of the structures that either have parameters
with names different from \tye{a}, \tye{b} or \tye{c} or that have an \tye{a},
\tye{b} or \tye{c} with a different type match \ty{B}.

As an example, all of the following lines define a structure that is an
instance of \ty{B}.
\begin{quote}
  \ins{c: "bottles of beer", a: 10}\\
  \ins{a: 10, b: 5 c: "bottles of beer"}\\
  \ins{a: 10, b: 5}
\end{quote}

\paragraph{Types of nested structures}
The statement
\begin{quote}
  \ty{C}\tye{ = \ins{a: B, b: String}}
\end{quote}
defines \ty{C} to be shorthand for the type of structures that have properties
with names \tye{aB} and \tye{aString}, with types \tye{B} and \tye{String}
respectively. Since we can further expand \ty{B}, this is in turn shorthand for
\begin{quote}
  \ins{a: \ins{a: Int, ?b: Int, ?c: String}, b: String}. 
\end{quote}

As an example, the following line defines a structure that is an instance of
\ty{C}.
\begin{quote}
  \ins{b: \ins{a: 10, b: 5}, a: "bottles of beer"}
\end{quote}

\paragraph{Union types}
The statement
\begin{quote}
  \ty{D}\tye{ = A | B}
\end{quote}
defines a type that contains both the structures contained in \tye{A} and the
structures contained in \tye{B}.

As an example, both of the following lines define a structure that is an
instance of \ty{D}.
\begin{quote}
  \ins{a: 10, b: tomatoes}\\
  \ins{a: 10, b: 5}
\end{quote}


\subsection{Structural type system}
This section discusses the structural type system of AiryScript by providing an
extended example.

We look at the type of object $o$ defined as
\begin{quote}
  \ins{a: 10, c: "bottles of beer"}.
\end{quote}
The type of $o$ is \ins{a: Int, c: String}. A discussion of the relation between
the type of $o$ and the following types should clarify how the type system of
AiryScript works.
\begin{quote}
  \ty{A}\tye{ = \ins{c: String, a: Int}}\\
  \ty{B}\tye{ = \ins{a: Int, ?b: Int, ?c: String}}\\
  \ty{C}\tye{ = \ins{a: Int, b: String, c: String}}\\
  \ty{D}\tye{ = \ins{c: Int, a: String}}
\end{quote}
\begin{itemize}
  \item $o$ is an instance of \ty{A}. In fact, \ty{A} is just shorthand for
    $o$’s type. The order in which \tye{c} and \tye{a} appear does not matter:
    as far as AiryScript is concerned the types \ins{c: String, a:Int}
    and \ins{a:Int, c: String} are identical. The one-dimensional nature of text
    makes it so that we have to specify an ordering that is actually
    meaningless.

  \item $o$ is an instance of \ty{B}, which also means that \ty{A} is a subtype
    of \ty{B}.  We can ‘ignore’ the optional properties of \tye{c}. Note that
    the fact that \ty{C} can be a super type of \ty{B} without having to specify
    this in \ty{C} is quite special, and typical for structural type systems.

  \item $o$ is not an instance of \ty{C}. \ty{A} and \ty{C} are not related.
    
    While in most existing structural type systems, like the one of Haxe, \ty{C}
    would be a subtype of \ty{B}, this is not the case in AiryScript. This is
    because AiryScript does not allow users to specify properties that are not
    defined by the type. \ins{a: "bla"} is not an instance of type \ins{a:
    String, b: String} in AiryScript. The only way to to create a typing
    hierarchy is through optional parameters.

  \item $o$ is not an instance of \ty{D}. \ty{A} and \ty{D} are not related.
    
    Even though \ty{D} also consists of two properties, one an \tye{Int} and one
    a \tye{String}, \ty{D} is unrelated to \ty{A} because the names of these
    properties are different in \ty{D} and \ty{A}. The names of properties are
    part of the type, which allows AiryScript to make types independent of the
    order of their properties.
\end{itemize}

\subsection{Operations}
\label{sec:operations}
\subsubsection{General types}
The type system of AiryScript is.
\begin{itemize}
  \item Strongly typed: when AiryScript expects a parameter of type $T$, it is
    not possible to provide it with another type. It is not possible to cast
    values to another type, and there is no type that matches all types.
  \item Statically typed: AiryScript determines and checks types at
    compile-time.
  \item Structural: type compatibility and equivalence are determined by the
    type’s actual structure or definition. Even if 
  \item Implicit: AiryScript never requires the user to explicitly declare any
      types.
\end{itemize}


\subsubsection{Basic types}
\begin{description}
  \item[\ty{Time}] \tye{= \{?h:Int (0), ?m:Int (0), ?s:Int (0)\}}

    Specifies either a time period (e.g. a flight takes 5 hours) or a time of
    day (e.g. a flight is at 13h).
  \item[\ty{Date}] \tye{= \{d:Int, m:Int, y:Int\}}

    Specifies a date.
  \item[\ty{DateTime}] \tye{= \{date:Date, ?time:Time (\{ \})\}}

    Specifies a specific point in time. 
  \item[\ty{Price}] \tye{= \{?dollar:Int (0), ?cent:Int (0)\}}
  \item[\ty{TimePeriod}] \tye{= \{from:DateTime, to:DateTime\}}
\end{description}

\subsubsection{City}
City types are
\begin{description}
  \item[\ty{City}] \tye{= \{?name: String\}}
  \item[\ty{City\_data}] \tye{= \{name: String\}}
\end{description}
City operations are


\subsubsection{Airport}
Airport types are
\begin{description}
  \item[\ty{Airport}] \tye{= \{?short:String,\\
      ?name: String,\\
      ?city:City\}}
  \item[\ty{Airport\_data}] \tye{= \{short:String,\\
      name: String,\\
      city:City\}}
\end{description}


\subsubsection{Distance}
Distance types are
\begin{description}
  \item[\ty{Dist}] \tye{= \{?from:City,\\
      ?to:City,\\
      ?dist:Int\}}
  \item[\ty{Dist\_data}] \tye{= \{from:City,\\
      to:City,\\
      dist:Int\}}
\end{description}

\subsubsection{Flight time}
Flight time types are
\begin{description}
  \item[\ty{FlightTime}] \tye{= \{?from:City,\\
      ?to:city,\\
      ?airplaneType:airplaneType,\\
      ?duration:Time\}}
  \item[\ty{FlightTime\_data}] \tye{= \{from:City,\\
      to:city,\\
      airplaneType:airplaneType,\\
      duration:Time\}}
\end{description}


\subsubsection{Airplane type}
Airplane type types are
\begin{description}
  \item[\ty{AirplaneType}] \tye{= \{?name:String\}}
  \item[\ty{AirplaneType\_data}] \tye{= \{name:String, arrangement:[Seat\_data]\}}
\end{description}


\subsubsection{Seats}
Seat types are
\begin{description}
  \item[\ty{Seat}] \tye{= \{?number:Int, ?type:String\}}
  \item[\ty{Seat\_data}] \tye{= \{?number:Int, ?amt:Int, type:String\}}
\end{description}

\subsubsection{Airline}
\begin{description}
  \item[\ty{Airline}] \tye{= \{?short:String, ?name:String\}}
  \item[\ty{Airline\_data}] \tye{= \{short:String, name:String\}}
\end{description}

\subsubsection{Templates}
\begin{description}
  \item[\ty{Template}] \tye{= \{?fln:String,\\
      ?airline:String,\\
      ?from:City,\\
      ?to:City,\\
      ?airplaneType:AirplaneType\}}
  \item[\ty{Template}] \tye{= \{fln:String,\\
      airline:String,\\
      from:City,\\
      to:City,\\
      airplaneType:AirplaneType,\\
      prices:[SeatInstance\_data],\\
      periods:[Period\_data]\}}
\end{description}

\subsubsection{Flights}
\begin{description}
  \item[Flight] \tye{= \{?template:Template,\\
      departure:DateTime,\\
      ?arrival:DateTime,\\
      ?airplaneType:AirplaneType,\\
      ?duringInterval:TimePeriod\}\\
      | \\
      \{?template:Template,\\
      ?departure:DateTime,\\
      arrival:DateTime,\\
      ?airplaneType:AirplaneType,\\
      ?duringInterval:TimePeriod\}\\
      | \\
      \{?template:Template,\\
      ?departure:DateTime,\\
      ?arrival:DateTime,\\
      ?airplaneType:AirplaneType,\\
      duringInterval:TimePeriod\}
    }
\end{description}

\subsubsection{Period}
\begin{description}
  \item[\ty{ContainedPeriod}] \tye{= \{?from:Data, ?to:Date, ?day:Date\}}
  \item[\ty{Period}] \tye{= \{?contained:ContainedPeriod,\\
      ?from:Date,\\
      ?to:Data,\\
      ?day:String,\\
      ?startTime:Time\}}
  \item[\ty{Period\_data}] \tye{= \{?from:Date,\\
      ?to:Data,\\
      ?weekday:String,\\
      startTime:Time\}}
\end{description}


\subsubsection{Seat instances}
\begin{description}
  \item[\ty{SeatInstance}] \tye{= \{number:Int,\\
      ?amt:Int\} \\
      | \\
      \{type: String\}}
  \item[\ty{SeatInstance\_data}]  \tye{= \{number:Int,\\
      ?amt:Int,\\
      price:Price\}\\
      | \\
      \{type:String,\\
      price:Price\}}
\end{description}

