%        File: solutions.tex
%     Created: ma okt 10 02:00  2011 C
% Last Change: ma okt 10 02:00  2011 C
%
%\documentclass[a4paper,10pt,notitlepage]{article}
\documentclass[a4paper,10pt,notitlepage]{scrartcl}
%\documentclass{beamer}
%\usetheme{Warsaw}

% Changed behavior
\usepackage[english]{babel}
\usepackage[usenames,dvipsnames,svgnames]{xcolor} % color ?
\usepackage{hyperref} % provides hyperref, url and href
\usepackage{graphicx} % color graphs ?
%\usepackage[justification=centering]{caption} % centers captions
\usepackage{mdframed} % framed boxes that span accross pages
  % Provides spacing option. Use \singlespacing, \doublespacing or
  % \onehalfspacing to control spacing, or use the doublespace, onehalfspace or
  % singlespace environments.
\usepackage{setspace}
\usepackage{enumitem}
\usepackage{dblfloatfix} %figure* in multicol can also be at bottom

% Commands
\usepackage{xifthen} 
\usepackage{xparse} % advanced environments
\usepackage[normalem]{ulem}  % strikeout using \sout in normal mode
\usepackage{cancel} % strikeout using \cancel in math mode
\usepackage{soul} % highlighting using \hl
%\usepackage[authoryear]{natbib}% provides citet, citep, sitet*
%\usepackage[backend=bibtex,style=alphabetic]{biblatex} 
%\addbibresource{library}%

% Environments
\usepackage{framed} % framed environment
\usepackage{verbatim} % something to do with the verbatim environment…
  \setlength{\columnsep}{0.5cm} % set column separation
\usepackage{multicol} % multi column layout
\usepackage{subfig}   % provides subfloat
\usepackage{algpseudocode}
\usepackage{algorithm}
\usepackage{pstricks}
\usepackage{tikz} % provides tikzpicture environment
  % Provides arrows and replacing functions
  \usetikzlibrary{decorations.pathreplacing,
    decorations.markings, % for postdecorating markings
    arrows,
    positioning,
    calc,
    shapes,
    scopes,
    backgrounds
  }
\usepackage{listings} % provides listings environment for code listing
  \lstset{%
    basicstyle=\small
}

% Math
\usepackage{amsmath,amssymb,amstext}
\usepackage{mathtools} % provides matrix* with alignment option!
% Provides theorems, …, & proofs
  \usepackage{amsthm}
    % Standard theorems numbered by section
    \newtheorem{theorem}{Theorem}[section]
    \newtheorem{lemma}[theorem]{Lemma}
    \newtheorem{proposition}[theorem]{Proposition}
    \newtheorem{corollary}[theorem]{Corollary}
    % Self-defined ‘plain theorems’; continuous numbering
    \newtheorem{ptheorem}{Theorem}
    \newtheorem{plemma}[ptheorem]{Lemma}
    \newtheorem{pproposition}[ptheorem]{Proposition}
    \newtheorem{pcorollary}[ptheorem]{Corollary}
    \theoremstyle{definition}
    \newtheorem{pdef}{Definition}
    %\theoremstyle{remark}
    \newtheorem{premark}{Remark}
    \renewcommand{\qedsymbol}{$∎$}


%%%%%%%%%%%%%%%%%%%%%%%%%%%%%%%%%%%%%%%%%%%%%%%%%%%%%%%%%%%%%%%%%%%%%%%%%%%%%%
                                  % UNICODE %
%%%%%%%%%%%%%%%%%%%%%%%%%%%%%%%%%%%%%%%%%%%%%%%%%%%%%%%%%%%%%%%%%%%%%%%%%%%%%%
\usepackage{fontspec}
%\usepackage{unicode-math}
%  \setmainfont{XITS}
%  \setmathfont{XITSMath}
%  \setmonofont[Scale=0.80]{DejaVu Sans Mono}
%  \setmonofont[Mapping=tex-text,Scale=0.80]{DejaVu Sans Mono}

  %\setmainfont{LMRoman10-Regular}
  %\setmainfont{LMSans10-Regular}
  %\setmathfont{LMMath-Regular}
  %\setmonofont{LMMono10-Regular}
% Provides listings environment (like verbatim but more powerful)

%%%%%%%%%%%%%%%%%%%%%%%%%%%%%%%%%%%%%%%%%%%%%%%%%%%%%%%%%%%%%%%%%%%%%%%%%%%%%%
                                % MY COMMANDS %
%%%%%%%%%%%%%%%%%%%%%%%%%%%%%%%%%%%%%%%%%%%%%%%%%%%%%%%%%%%%%%%%%%%%%%%%%%%%%%
% From http://www.dfcd.net/articles/latex/latex.html
\newcommand{\avg}[1]{\left< #1 \right>} % for average
\let\underdot=\d % rename builtin command \d{} to \underdot{}
\renewcommand{\d}[2]{\frac{d #1}{d #2}} % for derivatives
\newcommand{\dd}[2]{\frac{d^2 #1}{d #2^2}} % for double derivatives
\newcommand{\pd}[2]{\frac{\partial #1}{\partial #2}}  % partial derivative
\newcommand{\pdd}[2]{\frac{\partial^2 #1}{\partial #2^2}} % double ""
\newcommand{\abs}[1]{\left| #1 \right|} % absolute value
\let\baraccent=\= % rename builtin command \= to \baraccent
\renewcommand{\=}[1]{\stackrel{#1}{=}} % for putting numbers above =

\newcommand{\argmax}[1]{\underset{#1}{\operatorname{argmax}}}
%\newcommand{\argmax}{\operatornamewithlimits{argmax}}
%\DeclareMathOperator*{\argmax}{arg\,max}
\newcommand{\field}[1]{\mathbb{#1}} % requires amsfonts
\newcommand{\jdef}[1]{{\bfseries \color{MidnightBlue}{#1}}}
\newcommand{\je}[1]{{\bfseries \color{MidnightBlue}{#1}}}
\newcommand{\jst}[2]{\ensuremath{#1^{(#2)}}}

%%%%%%%%%%%%%%%%%%%%%%%%%%%%%%%%%%%%%%%%%%%%%%%%%%%%%%%%%%%%%%%%%%%%%%%%%%%%%%
                              % MY ENVIRONMENTS %
%%%%%%%%%%%%%%%%%%%%%%%%%%%%%%%%%%%%%%%%%%%%%%%%%%%%%%%%%%%%%%%%%%%%%%%%%%%%%%
% fmpage
\newsavebox{\fmbox}
\NewDocumentEnvironment{fmpage}{O{\textwidth}}
  {\begin{mdframed}}
  {\end{mdframed}}
% jinfo
\NewDocumentEnvironment{jinfo}{O{\textwidth}o} %%%%%%%%
  {\begin{fmpage}[#1]}
  {\end{fmpage}
    \IfNoValueTF{#2}
    {} %empty
    {\begin{center}Source: \url{#2}\end{center}} %not empty
  }
% jpic
\newenvironment{jcpic}[1][] %%%%%%%%%
  {\begin{center}\begin{tikzpicture}[#1]}
  {\end{tikzpicture}\end{center}}
% tablehere
\makeatletter
\newenvironment{tablehere}
  {\def\@captype{table}}
  {}
% figurehere
\newenvironment{figurehere}
  {\def\@captype{figure}}
  {}
\makeatother

%%%%%%%%%%%%%%%%%%%%%%%%%%%%%%%%%%%%%%%%%%%%%%%%%%%%%%%%%%%%%%%%%%%%%%%%%%%%%%
                                 % AESTATICS %
%%%%%%%%%%%%%%%%%%%%%%%%%%%%%%%%%%%%%%%%%%%%%%%%%%%%%%%%%%%%%%%%%%%%%%%%%%%%%%
\usepackage[a4paper,hmargin=2.5cm,vmargin=2.5cm]{geometry}
%\usepackage[calcwidth]{titlesec}
%\titleformat{\section}{\large\bfseries}{\thesection.}{0.5em}{}
%\titleformat{\section}{\large\bfseries}{§}{0.5em}{}
%\titleformat{\subsection}{\normalsize\bfseries}{}{0em}{}
%\titleformat{\subsubsection}[block]
%  {\normalsize\it}
%  {}
%  {0em}
%  {}
%  [\vspace{-1em}\rule{\titlewidth}{1pt}]
%\renewcommand{\bar}{\overline}
\let\vaccent=\v % rename builtin command \v{} to \vaccent{}
\renewcommand{\v}[1]{\ensuremath{\mathbf{#1}}} % for vectors

%Fancy headers
\usepackage{fancyhdr}
  \usepackage{lastpage} % the \thepage{} command, total #pages
  \setlength{\headheight}{15pt} 
  \pagestyle{fancyplain}
    %\renewcommand{\chaptermark}[1]{\markboth{#1}{}}
    \lhead{Designing AiryScript}
    \chead{}
    \rhead{\fancyplain{}{David Merckx, Jonathan Merlevede, Tom op’t Roodt,
    Kristof Peeters}}
    %\rhead{\fancyplain{}{\textit{\leftmark}}}
    \lfoot{}
    \cfoot{\fancyplain{}{\thepage{} of \pageref{LastPage}}}
    \rfoot{}
\usepackage[calcwidth]{titlesec}
%\titleformat{\section}{\large\bfseries}{\thesection.}{0.5em}{}

%%%%%%%%%%%%%%%%%%%%%%%%%%%%%%%%%%%%%%%%%%%%%%%%%%%%%%%%%%%%%%%%%%%%%%%%%%%%%%
                             % DOCUMENT SPECIFIC %
%%%%%%%%%%%%%%%%%%%%%%%%%%%%%%%%%%%%%%%%%%%%%%%%%%%%%%%%%%%%%%%%%%%%%%%%%%%%%%
% Document-specific commands
  \hyphenation{Handels-reizigers-probleem}


\begin{document}
\title{Designing AiryScript}
\subtitle{On designing AiryScript, a DSL developed by order of GlobAir Inc. for
populating its database.}
\author{David Merckx\\Jonathan Merlevede\\Tom op’t Roodt\\Kristof Peeters}
%\date{\datenow}%
% ----------------------------------------------------------------
\maketitle
%{\bf
%  \noindent
%}
\begin{abstract}
  \bf
  \noindent
  GlobAir Inc. is a multinational company catering to airline companies that
  keeps track of airline connections all over the world. This document documents
  AiryScript, a DSL (Domain Specific Language) that facilitates populating
  GlobAir Inc.’s airline database.\footnotemark{} It also documents the problem domain
  underlying AiryScript and the decisions made during its development.
  \footnotetext{
    GlobAir Inc. is a fictional company and has nothing to with the Globair
    Group or any other company that actually exists. This is a document made by
    students of the University of Leuven for the course Comparative Programming
    Languages taught by Dave Clarke.
  }
\end{abstract}
\thispagestyle{fancyplain}
\section{Problem Domain}
This section describes the problem domain underlying AiryScript. The first
subsection lists domain elements already present in the original commissioning
made by GlobAir Inc.. The second subsection continues by filling in the blanks
that were left in there. Together, the first and second subsections provide a
complete description of the problem domain. The third subsection analyses the
problem domain, trying to distinguish core concepts from details and connecting
the concepts. It thereby acts as an important basis for the design of
AiryScript.

\subsection{Original commissioning}
This subsection documents the domain elements present in the original
commissioning made by GlobAir Inc..


\subsection{Filling in the blanks}
This subsection documents problem domain elements that are not in the original
commissioning made by GlobAir Inc.. Most of these decisions were made in
consultation with GlobAir’s domain expert Dave Clarke.

\subsection{Domain analysis}
This subsection analyses the domain elements introduced by the previous two
subsections. It acts as a basis for decisions on the syntax of AiryScript.


\section{Problem Domain}
This section describes the problem domain underlying AiryScript. The first
subsection lists domain elements already present in the original commissioning
made by GlobAir Inc.. The second subsection continues by filling in the blanks
that were left in there. Together, the first and second subsections provide a
complete description of the problem domain. The third subsection analyses the
problem domain, trying to distinguish core concepts from details and connecting
the concepts. It thereby acts as an important basis for the design of
AiryScript.

\subsection{Original commissioning}
This subsection documents the domain elements present in the original
commissioning made by GlobAir Inc..


\subsection{Filling in the blanks}
This subsection documents problem domain elements that are not in the original
commissioning made by GlobAir Inc.. Most of these decisions were made in
consultation with GlobAir’s domain expert Dave Clarke.

\subsection{Domain analysis}
This subsection analyses the domain elements introduced by the previous two
subsections. It acts as a basis for decisions on the syntax of AiryScript.


\section{Design description}
\subsection{Picking a programming language}

\subsection{Devising a language syntax}

\subsubsection{Iteration one}

\subsubsection{Iteration two}

\subsubsection{Iteration three}

\subsection{Defining appropriate types}


\section{Implementation}

\subsection{Picking a programming language}
We decided to program AiryScript in Scala fairly early on for the following
reasons.
\begin{itemize}
  \item We wanted to be able to really define our own language syntax.
    Generally, this means having our own parser and not merely calling 'eval' in
    some other language. Also, all of the members of our team prefer typed
    languages. Therefore, we also wanted to create a typed language. We could
    not see how to do this without making our own parser. We were already in
    possession of some basic parser code for Scala.
  \item Some of the members of our team were already familiar with Scala.
  \item The whole team has a thorough knowledge in Java gained in other courses
  and projects. This smooths the learning curve of learning (the basic
  principles of) an almost entirely new language. 
  \item Scala has very good support for pattern matching through case classes
    (i.e. tagged unions). We make extensive use of this in our code.
  \item Some of the members of our team had been looking for a good opportunity
    to get to know Scala for some time.
\end{itemize}


\subsection{Implementation overview}

We divided the implementation in 3 parts:
\begin{enumerate}
\item The parser: Responsible for parsing a string into the correct operation,
or returning a parse error if the given expression was incorrect.
\item The operation handler: This component contains the implementation logic
of all the operators. The handler is responsible for the consistency checking of
all the provided arguments of an operation and either throwing an error if one
or more arguments were incorrect/incomplete or performing the operation. The
database will be accessed for verification as well as performing the actual
\textsc{crud} operation.
\item The database: We used an SQL database to persist. The
implementation of this database is closely related to the domain model.
\end{enumerate} 
	
\par
All of the operations and types are represented by a single case class in Scala,
which captures all of the required information to use this. All three of the
aforementioned components are united in a single script which will execute them
in the following order. First, the parser will parse the user's input into an
operation, after which this is passed to the second phase. The handler will
subssequently handle these operations by pattern matching over the different
kinds of operators and their respective arguments. In case an error occured
somewhere along this execution path, the user will be informed of what went wrong.


\subsection{Scala Parser}

The responsibility of our parser is to turn a string into one of the possible
operations, or throw an error if the syntax is incorrect. We extended the
\class{StandardTokenParsers} class available in the standard scala api, this
allowed to make a clean, extendable parser written in pure functional code.

We mentioned the 

\subsubsection{Tokens} 

The first step in the process of parsing is translating the string into a list
of tokens. Special sequences within the string will be converted into special
tokens aswell.

\par
There are 4 kinds of tokens:
\begin{description} 
\item[keywords] There are 2 sort of sequences that can be parsed into a keyword
token:
	\begin{description}
	\item[delimiters] whenever one of these sequences is recognized it is parsed
	into a delimiter-token. The delimiters in our implementation are: \sn{\{},
	\sn{\}}, \sn{,}, \sn{:}.
	\item[reserved keywords] These sequences are only parsed into their
	corresponding tokens if they are not part of a larger word. Concretely this
	means that they are only parsed if they are surrounded by white space or
	delimiters. The reserved keywords in our implementation are all the words used
	to describe operations: \sn{ADD}, \sn{REMOVE}, \sn{CITY}, \sn{AIRPORT}, etc.
	\end{description}
\item[numeric literal] Whenever a sequence of numbers is spotted, this will be
parsed into a numeric literal token.
\item[string literal] Whenever a sequence between quotation marks is found it is
converted as a whole into a string literal, so within this sequence no further
splitting is done. This is usefull for providing strings that contain spaces
such as \sn{New York}.
\item[identifier] Basically any kind of words that were not described before,
they are sepeterated by delimiters and white space.
\end{description}
	
\par
Parsing the string \sn{ADD CITY \{name:Brussels, short:BRU\}} will result into
the following list: \sn{keyw ADD}, \sn{keyw CITY}, \sn{"keyw \{"}, \sn{ident
name}, \sn{keyw :}, \sn{ident Brussels}, \sn{keyw ,}, \sn{ident short},
\sn{keyw  :}, \sn{keyw BRU}, \sn{keyw \}}.
 
\par
Note that the actual creation of these tokens is completely done by the
\class{StdTokenParsers} class that we extended. The only thing we provide is the
list with delimiters and reserved keywords.


\subsubsection{Parsing operators}

Ones a list of tokens is obtained the actual parsing can start. We can use the
syntax provided by the StdTokenParsers class to do this.

\par
The first step in parsing an operation is quite simple. Since every operator
has a unique syntax to call it we just try to match the current string with
a list of all the possible operators. A fragment of this part:

\begin{lstlisting}
def parseOp(): Parser[Operation]  =
    "ADD" ~> "CITY" ~> parseCityData ^^ {c => AddCity(c)} |
    "ADD" ~> "AIRPORT" ~> parseAirportData ^^ {a => AddAirport(a)}
    //...
\end{lstlisting}

\par
The \class{StanderdTokensParsers} provides us with a range of special operations
that can easily be used to parse tokens. The \sn{\textasciitilde>} in the
fragment above is used to match a keyword token corresponding to that string,
the result obtained form this match is however discarted. The
\sn{\textasciicircum\textasciicircum} is used to retain the result from this
line of match operators. 

\par
The \sn{parseCityData} in the code above is simple a val which contains the
definition to parse a \class{City\_data}. If the provided string actually
matches with this definition then an actual \class{City\_data} object will be
constructed and passed to the function after the \sn{\textasciicircum\textasciicircum}
operator. We use this to construct a new \class{AddCity} case class, which
maps 1-to-1 with the definition of add city as explained in
\ref{sec:operations}.

\par
The following string will match the first operator in the definition above, the
definition in \sn{parseCityData} will create the actual city with name
\sn{"Brussels"} which is stored inside a new \class{AddCity}.

\begin{lstlisting}
"ADD CITY {name:Brussels}"
\end{lstlisting}


\subsubsection{Parsing types}

Parsing types is analoug to parsing operators, only that a few additional
problems have to be tackled here. Let us look at the example of parsing city
data. For this example sn{\{name:Brussels, short:BRU\}} and \sn{\{short:BRU,
name:Brussels\}} are both valid descriptions of city data. The order in which
the attributes appear in the definition do not matter. Also the user does not
have to explicitly mention what kind data he is providing. That means that we
need to check if the data he provided matches any of the possible formats for
this operation. Since data types can be nested we need to check all
possibilities and backtrack when a mismatch is spotted.

\par
Luckily the implementation for backtracking is already present in the parser we
extended. We have to use the right operators however to trigger this. Whenever
we use the \sn{\textasciicircum?} operator it will behave the same as a the
\sn{\textasciicircum\textasciicircum} operator except that when the result cannot
match we will backtrack instead of fail parsing.

\par
We implemented a general way of dealing with the order of attributes. Whenever
we start parsing any kind of data object we pull out all of its attributes and
place them in a map. This map can thereafter be consulted to see if all required
attributes were provided, regardless of the order they were originally provided
in.

\subsection{Operation handler} 
\subsection{SQL Database}

\section{Tutorial}


\end{document}
