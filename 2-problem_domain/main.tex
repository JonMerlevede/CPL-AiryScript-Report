\section{Problem Domain}
\label{sec:domain}
%A thorough understanding of the problem domain is not only crucial to the
%development of a decent DSL, but also for its understanding.

This section describes the problem domain underlying AiryScript. It does not
contain the elements that were already present in the original commissioning
made by GlobAir Inc..

Subsection \ref{sec:domain_description} documents the domain elements that are
missing from the original description.
%
Subsection \ref{sec:domain_analysis} contains an analysis of the entire problem
domain. It lists all the relations that exist in between domain elements.
It thereby acts as an important basis for the design of AiryScript.

\subsection{Domain description}
\label{sec:domain_description}
This subsection documents all the additional elements that are missing from the
original job description made by GlobAir Inc.. It elaborates and justifies all
elements that do not have domain expert David Clarke as their source.


\subsubsection{Unsupported operations}
This subsection gives an overview of some of the more important operations that
are currently not supported by AiryScript.

There is some functionality that might be required of AiryScript before it goes
live. AiryScript does currently not provide the following functionality because
AiryScript is still just a proof of concept, and the assignment does not mention
it.
\begin{itemize}
  \item Booking flights.
  \item Retrieving information from the database.
  \item Keeping track of individual airplanes.
\end{itemize}

In consultation with the domain expert, AiryScript does currently not provide
the following functionality.
\begin{itemize}
  \item Overbooking, i.e. the number of seats in an airplane has to be equal to
    the number of seats the user has to specify prices for.
  \item Codeshare agreements, i.e. a flight has to map onto a single airline
    company.
\end{itemize}

\subsubsection{Unspecified, supported operations}
In consultation with the domain expert, AiryScript provides the following
functionality that is not specified in the original assignment.
\begin{itemize}
  \item Changing the airplane type of flight templates. By changing the airplane
    type of flight templates, the history track of flown flights is not changed,
    and neither are any of the specific instances of the flight templates for
    which another airplane was already specified.

    AiryScript also supports changing the airplane type of specific flights. It
    does currently not support specifying periods over which a flight template
    is serviced by a specific airplane type.
\end{itemize}


\subsubsection{Additional decisions}
When implementing AiryScript, we also made some decisions that we did not
discuss with the domain expert.
\begin{itemize}

  \item The original domain description is not specific on how elaborate
    descriptions of new flights can be. The assignment describes flight
    templates, which correspond to weekly recurring flights. The assignment also
    hints that there can also be individual, non-recurring flights.
    
    AiryScript supports defining single flights and weekly flight patterns,
    possibly with multiple flying days per week and multiple flights per day. It
    also supports specifying multiple, possibly disconnected time periods over
    which the flight pattern is valid.
    
    This means that users can make statements like the following.
    \begin{quote}
      There is a flight every Monday at 13:00 and Friday at 17:30 starting from
      August 2013 and ending at September 2013. Starting from October 2013 and
      until the end of times, only the flight on Monday remains.
    \end{quote}
    This also means that users cannot specify a flight pattern occurring every
    third Friday of the month.
    
    AiryScript supports manually adding flights to flight codes. In this
    way, exceptional one-time only flights do not require modifying the flight
    template that corresponds to the flight code. Also users who really want to
    schedule flights each third Friday of the month can use this functionality
    to manually add flights.

  \item AiryScript supports changing the prices of individual seats of
    individual flights and the prices of flight templates, possibly only over a
    specified period. Users can also use seat types to specify groups of seats
    for which to change the price; these seat types are specified by the type of
    airplane.

    Binding seat types to airplanes has advantages and disadvantages. The
    greatest advantage is that users can use seat types to specify prices on a
    newly created flight, without having to specify how many seats seats should
    cost extra because they come with extra leg space. The greatest disadvantage
    is that it currently makes it impossible to specify additional pricing
    groups that are valid only for a specific flight.

    We feel that the advantages of binding seat types to airplane types outweigh
    its disadvantages. Users can still specify any prices they want, because
    they can change prices of individual seats. If required, it would also
    certainly be possible to provide names for groups of seats that are tied to
    individual flights or flight templates.

    AiryScript does not support changing prices as a function of time.

  \item AiryScript always requires users to specify an airplane type when
    creating individual flights or flight templates.

    We believe this restriction is justified, because AiryScript supports
    changing the airplane type of a flight after its creation. If a user really
    has no idea of the type of airplane that is used for the flight he is
    inserting, he or she can create a dummy airplane.
\end{itemize}

One of the more confusing aspects of the domain description is the description
of a flight. When the domain expert refers to a flight it can mean either an
individual flight or a recurring flight. To avoid confusion we use flight only
in its first meaning. We use  \jdef{flight template} instead of flight in its
second meaning.

\subsection{Domain analysis}
\label{sec:domain_analysis}
This subsection analyses the domain elements introduced by the previous two
subsections. It acts as a basis for decisions on the syntax of AiryScript.

