\section{Types and operations}
\label{sec:operations}

\subsection{Basic types}
\begin{description}
  \item[\ty{Time}] \tye{= \{?h:Int, ?m:Int, ?s:Int\}}

    Specifies either a time period or point in time where \tye{h} stands for the
    number of hours, \tye{m} for the number of minutes and \tye{s} for the
    number of seconds. The default value of all properties is zero.
  \item[\ty{Date}] \tye{= \{d:Int, m:Int, y:Int\}}

    Specifies a date. \tye{d} stands for the day, \tye{m} for the month and
    \tye{y} for the year.
  \item[\ty{DateTime}] \tye{= \{date:Date, ?time:Time\}}

    Specifies a specific point in time. The default value for the time is at
    midnight.
  \item[\ty{Price}] \tye{= \{?dollar:Int, ?cent:Int\} | \{?euro:Int, ?cent:Int\}}

    Specifies an amount of money in either dollar or euros. The default value of
    all properties is zero.
  \item[\ty{TimePeriod}] \tye{= \{from:DateTime, to:DateTime\}}

    Specifies the time period in between the points in time defined by
    \tye{from} and \tye{to}.
\end{description}

\subsection{City}
\subsubsection{Types}
City types are
\begin{description}
  \item[\ty{City}] \tye{= \{?name: String\}}
  \item[\ty{City\_data}] \tye{= \{name: String\}}
\end{description}
where \tye{name} specifies the name of the city. 

\subsubsection{Operations}
City operations are
\paragraph{Adding a city}
\begin{quote}
  \lstinline{ADD CITY <data:City_data>}
\end{quote}
Adds the city specified by \tye{data}.

\begin{texa}[add the city of New York.]
  \lstinline|ADD CITY {name: "New York"}|
\end{texa}

\paragraph{Changing a city}
  \begin{quote}
    \lstinline|CHANGE CITY <citySelector:City> TO <cityChange:City>|
  \end{quote}
Changes the city or cities specified by \tye{citySelector} to have the values
specified by \tye{cityChange}.

\begin{texa}[change the name of the city of New York to York.]
  \lstinline|CHANGE CITY { name: "New York" } TO { name: "York" }|
\end{texa}

\paragraph{Removing a city}
\begin{quote}
  \lstinline{REMOVE CITY <selector:City>}
\end{quote}
Removes the city or cities specified by \tye{selector}.

\begin{texa}[remove the city York.]
  \lstinline|REMOVE CITY {name: "York"}|
\end{texa}


\subsection{Airport}
\subsubsection{Types}
Airport types are
\begin{description}
  \item[\ty{Airport}] \tye{= \{?short:String,\\
      ?name: String,\\
      ?city:City\}}
  \item[\ty{Airport\_data}] \tye{= \{short:String,\\
      name: String,\\
      city:City\}}
\end{description}
where
\begin{itemize}
  \item \tye{short} stands for the short code airlines use internally to
    represent airports.
  \item \tye{name} stands for the full-length name of the airport.
  \item \tye{city} stands for the city the airport is situated in.
\end{itemize}
\subsubsection{Operations}
Airport operations are
\paragraph{Adding an airport}
\begin{quote}
  \lstinline{ADD AIRPORT <data:Airport\_data>}
\end{quote}
Adds the airport specified by \tye{data}.
\begin{texa}[add JFK Airport to the city of New York.]
  \lstinline|ADD AIRPORT {short: JFK, name:"JFK Airport", city: {name: "York"}}|
\end{texa}

\paragraph{Changing an airport}
\begin{quote}
  \lstinline{CHANGE AIRPORT <selector:Airport> TO <change:Airport>}
\end{quote}
Changes the airport or airports specified by \tye{selector} to have the values
specified by
\tye{change}.
\begin{texa}[change JFK Airport to be in the city of York.]
  \lstinline|CHANGE AIRPORT {short: JFK} TO {city: {name: "York"}}|
\end{texa}

\paragraph{Removing an airport}
\begin{quote}
  \lstinline{REMOVE AIRPORT <Airport>}
\end{quote}
Removes the airport or airports specified by \tye{selector}.
\begin{texa}[remove JFK Airport.]
  \lstinline|REMOVE AIRPORT {short: JFK}|
\end{texa}

\subsection{Distance}
\subsubsection{Types}
Distance types are
\begin{description}
  \item[\ty{Distance}] \tye{= \{?from:Airport,\\
      ?to:Airport,\\
      ?dist:Int\}}
  \item[\ty{Distance\_data}] \tye{= \{from:Airport,\\
      to:Airport,\\
      dist:Int\}}
\end{description}
where \tye{dist} is the distance of the flight trajectory from \tye{from} to
\tye{to}.
\subsubsection{Operations}
Distance operations are as follows.

\paragraph{Adding a distance}
\begin{quote}
  \lstinline{ADD DISTANCE <data:Distance_data>}
\end{quote}
Adds the distance specified by \tye{data}.
\paragraph{Changing a distance}
\begin{quote}
  \lstinline{CHANGE DISTANCE <selector:Distance> TO <change:Distance>}
\end{quote}
Changes the distances specified by \tye{selector} to have the values specified
by \tye{change}.
\paragraph{Removing a distance}
\begin{quote}
  \lstinline{REMOVE DISTANCE <selector:Distance>}
\end{quote}
Removes the distance specified by \tye{selector}.

\subsection{Flight time}
\subsubsection{Types}
Flight time types are
\begin{description}
  \item[\ty{FlightTime}] \tye{= \{?from:Airport,\\
      ?to:Airport,\\
      ?airplaneType:airplaneType,\\
      ?time:Time\}}
  \item[\ty{FlightTime\_data}] \tye{= \{from:Airport,\\
      to:Airport,\\
      airplaneType:airplaneType,\\
      time:Time\}}
\end{description}
where \tye{time} is the time it takes an airplane of type \tye{airplaneType}
to travel from \tye{from} to \tye{to}.

\subsubsection{Operations}
Flight time operations are as follows.

\paragraph{Adding a flight time}
\begin{quote}
  \lstinline{ADD FLIGHT TIME <data:FlightTime_data>}
\end{quote}
Adds the flight time specified by \tye{data}.

\paragraph{Changing a flight time}
\begin{quote}
  \lstinline{CHANGE FLIGHT TIME <selector:FlightTime> TO <change:FlightTime>}
\end{quote}
Changes the flight time or flight times specified by \tye{selector} to have the
values specified by \tye{change}.

\paragraph{Removing a flight time}
\begin{quote}
  \lstinline{REMOVE FLIGHT TIME <selector:FlightTime>}
\end{quote}
Removes the flight time or flight times specified by \tye{selector}.


\subsection{Airplane type}
\subsubsection{Types}
Airplane type types are
\begin{description}
  \item[\ty{AirplaneType}] \tye{= \{?name:String\}}
  \item[\ty{AirplaneType\_data}] \tye{= \{name:String\}}
\end{description}
where \tye{name} is the name of the airplane type specified by an instance of

There are also seats in airplanes. Seat types are
\begin{description}
  \item[\ty{SeatType}] \tye{= String}
  \item[\ty{Seat}] \tye{= \{?number:Int, ?type:SeatType\}}
  \item[\ty{Seat\_remove}] \tye{= \{?number:Int, ?amt:Int, ?type:SeatType\}}
  \item[\ty{Seat\_data}] \tye{= \{?number:Int, ?amt:Int, type:SeatType\}}
\end{description}
where \ty{SeatType} represents a seat type and a \ty{Seat} represents a seat in
an airplane type.

\subsubsection{Operations}
Seat numbers have to be unique within an airplane type, but do not have to be
continuous (i.e. there can be a seat with number $n$ even if not all the seats
from 1 to $n-1$ exist). Overlapping seat numbers will cause any of the
operations below to throw an error.

\paragraph{Adding an airplane type}
\begin{quote}
\begin{lstlisting}
ADD AIRPLANE TYPE <data:AirplaneType_data>
WITH SEATS <seats:[Seat_data]>
\end{lstlisting}
\end{quote}
Adds the airplane type specified by \tye{data} with the seat arrangement
specified in \tye{seats}. For each element in \ty{Seat\_data}, AiryScript adds
the amount of seats specified by \tye{amt} of the type specified by \tye{type},
starting numbering at the seat number specified by \tye{number}.  The default
value for \tye{amt} is one, the default value of \tye{number} is equal to the
highest seat number so far plus one. AiryScript processes the seats in the order
they appear in the \tye{seats} list.


\begin{texa}[
Adds an airplane type with the name ‘Boeing 007’.
The airplane type has 100 seats: 10 first class seats, 20 business seats
and 70 economy seats. We assume these seat types already exist. The first class
seats are numbered from 1 to 10 (inclusive), the business seats are numbered 51
to 70 (inclusive) and the economy class seats are numbered from 11 to 50
(inclusive) and 71 to 100 (inclusive).
  ]
  {
\begin{lstlisting}
ADD AIRPLANE TYPE {
    name: "Boeing 007",
} WITH SEATS [
    {type: first, amt: 10},
    {type: business, number:51, amt: 20},
    {type: economy, amt: 40},
    {type: economy, amt: 30}
]
\end{lstlisting}
  }
\end{texa}

\paragraph{Adding seats to an airplane type}
\begin{quote}
  \begin{lstlisting}
ADD SEATS <datas:[Seat_data]>
TO AIRPLANE TYPE <airplaneSelector:AirplaneType>
  \end{lstlisting}
\end{quote}
Adds all of the seats specified by \tye{datas} to the airplane or airplanes
matched by \tye{airplaneSelector}.

\paragraph{Changing an airplane type}
\begin{quote}
  \lstinline|CHANGE AIRPLANE TYPE <selector:AirplaneType> TO <change:AirplaneType>|
\end{quote}
Changes the airplane type or airplane types specified by \tye{selector} to have
the values specified by \tye{change}.
%\begin{texa}[Changes the name of the ‘Boeing 007’ airplane to ‘Boeing 666’.]
%  {
%    \begin{lstlisting}
%CHANGE AIRPLANE TYPE {
%  name: "Boeing 007"
%} TO {
%  name: "Boeing 666"
%}
%      \end{lstlisting}
%  }
%\end{texa}

\paragraph{Changing seats of an airplane type}
\begin{quote}
  \begin{lstlisting}
CHANGE SEATS <seatSelectors:[Seat]>
OF AIRPLANE TYPE <airplaneSelector:AirplaneType>
TO <change:Seat>
  \end{lstlisting}
\end{quote}
Changes the seat or seats selected that match any of the seat selectors in
\tye{seatSelectors} and the airplane selector \tye{airplaneSelector} to have the
values specified in \tye{change}.
%\begin{itemize}
%  \item If any of the seat selectors in \tye{seatSelector} specifies a seat
%    number $n$, that seat selector selects the (single) seat with number $n$,
%    unless it specifies a seat type that does not match the type of seat $n$.
%  \item
If \tye{change} specifies a seat number $n$ and $m$ seats are selected,
the selected seat with the lowest number gets seat number $n$, the selected
seat with the second lowest number gets seat number $n+1$ and so on. If any
of the seat numbers in the range $[n,n+m-1]$ are already taken, the
operation is unsuccesful and does not change any seats.
%\end{itemize}

Note that the change seats operation is currently capable of merging seat types,
but not of changing the number of seats in seat types. This effect can be
achieved by adding and removing seats or by redefining the entire seat
arrangement of the airplane type.

\begin{texa}[Changes the Boeing 007 from ‘Adding an airplane type’ to have three
  more business seats: seats number one and three (which used to be first class
seats) and seat number 100 (which used to be an economy seat)]
  {
  \begin{lstlisting}
CHANGE SEATS [
  {number: 1}, {number: 3}, {number: 100}
] OF AIRPLANE TYPE {
  name: "Boeing 007"
} TO {
  type: business
}
  \end{lstlisting}
  }
\end{texa}

\paragraph{Changing all the seats of an airplane type}
\begin{quote}
  \begin{lstlisting}
CHANGE SEATS OF AIRPLANE TYPE <airplaneSelector:AirplaneType>
TO <data:[Seat_data]>
  \end{lstlisting}
\end{quote}
Changes the seats of the airplane type or airplane types selected by
\tye{airplaneSelector} to the arrangement specified by \tye{data}. Data is
processed like when first creating the airplane type.

This operation is syntactic sugar for first removing all the seats of the
airplane type, and then adding seats using \tye{data}.

\paragraph{Removing an airplane type}
\begin{quote}
  \lstinline|REMOVE AIRPLANE TYPE <selector:AirplaneType>|
\end{quote}
Removes the airplane type(s) specified by \tye{selector}. 

\paragraph{Removing seats from an airplane type}
\begin{quote}
  \begin{lstlisting}
REMOVE SEAT <seatSelector:Seat_remove>
FROM AIRPLANE TYPE <airplaneSelector:AirplaneType>
  \end{lstlisting}
\end{quote}
Removes the seat(s) matched by \tye{seatSelector} from the airplane type(s)
matched by \tye{airplaneSelector}.

\tye{amt} specifies the maximum number of seats to remove. If \tye{number} is
not specified but \tye{amt} is, AiryScript deletes the seats with the highest
seat numbers that match \tye{seatSelector}.

\begin{texa}[Removes ten business seats from the Boeing 007 as it is defined in
    ‘Adding an airplane type’. After this operation, the Boeing still contains
    first class seats numbered 1 to 10 and economy seats numbered 11 to 50 and
    71 to 100, but only the business seats that are numbered 51 to 60 remain;
  the seats numbered 61 to 70 are gone.]
  {
  \begin{lstlisting}
REMOVE SEATS [
  {amt: 10, type: economy}
] FROM AIRPLANE TYPE {
  name: "Boeing 007"
}
\end{lstlisting}
  }
\end{texa}


\subsection{Seats}
\subsubsection{Types}
\subsubsection{Operations}
\paragraph{Adding seats} (to an airplane type)
\begin{quote}
  \tye{CHANGE AIRPLANE TYPE <AirplaneType> ADD SEAT <Seat\_data>}\\
  \tye{CHANGE AIRPLANE TYPE <AirplaneType> ADD SEATS [<Seat\_data>]}
\end{quote}
\paragraph{Changing seats} (of an airplane type)
\begin{quote}
  \tye{CHANGE AIRPLANE TYPE <AirplaneType> CHANGE SEAT <Seat> TO <Seat\_data>}\\
  \tye{CHANGE AIRPLANE TYPE <AirplaneType> CHANGE SEATS [<Seat>] TO <Seat\_data>}
\end{quote}
\paragraph{Changing all seats} (of an airplane type)
\begin{quote}
  \tye{CHANGE AIRPLANE TYPE <AirplaneType> CHANGE SEATS TO [<Seat\_data>]}
\end{quote}
\paragraph{Removing seats} (from an airplane type)
\begin{quote}
  \tye{CHANGE AIRPLANE TYPE <AirplaneType> REMOVE SEAT <Seat>}\\
  \tye{CHANGE AIRPLANE TYPE <AirplaneType> REMOVE SEATS [<Seat>]}
\end{quote}



\subsection{Seat types}
Seat types correspond simply to string, which is the name of the seat type.
Still, users need to add seat types before they can be used.

\paragraph{Adding a seat type}
\begin{quote}
  \tye{ADD SEAT TYPE <String>}
\end{quote}
\paragraph{Changing a seat type}
\begin{quote}
  \tye{ADD SEAT TYPE <String>}
\end{quote}
\paragraph{Removing a seat type}
\begin{quote}
  \tye{REMOVE SEAT TYPE <String>}
\end{quote}

\subsection{Airline}
\subsubsection{Types}
\begin{description}
  \item[\ty{Airline}] \tye{= \{?short:String, ?name:String\}}
  \item[\ty{Airline\_data}] \tye{= \{short:String, name:String\}}
\end{description}

\subsubsection{Operations}
Flight time operations are
\paragraph{Adding an airline}
\begin{quote}
  \tye{ADD AIRLINE <Airline\_data>}
\end{quote}
\paragraph{Changing an airline}
\begin{quote}
  \tye{CHANGE AIRLINE <Airline> TO <Airline>}
\end{quote}
\paragraph{Removing an airline}
\begin{quote}
  \tye{REMOVE AIRLINE <Airline>}
\end{quote}



\subsection{Templates}
\subsubsection{Types}
Template types are
\begin{description}
  \item[\ty{Template}] \tye{= \{?fln:String,\\
      ?airline:String,\\
      ?from:City,\\
      ?to:City,\\
      ?airplaneType:AirplaneType\}}
  \item[\ty{Template}] \tye{= \{fln:String,\\
      airline:String,\\
      from:City,\\
      to:City,\\
      airplaneType:AirplaneType,\\
      prices:[SeatInstance\_data],\\
      periods:[Period\_data]\}}
\end{description}

\subsubsection{Operations}
Template operations are
\paragraph{Adding a template}
\begin{quote}
  \lstinline[language=airyscript]{ADD TEMPLATE <Template_data>}
\end{quote}
\paragraph{Changing a template}
\begin{quote}
  \tye{CHANGE TEMPLATE <Template> TO <Template>}
\end{quote}
\paragraph{Removing a template}
\begin{quote}
  \tye{REMOVE TEMPLATE <Template>}
\end{quote}

\subsection{Flights}
\subsubsection{Types}
\begin{description}
  \item[Flight] \tye{= \{?template:Template,\\
      departure:DateTime,\\
      ?arrival:DateTime,\\
      ?airplaneType:AirplaneType,\\
      ?duringInterval:TimePeriod\}\\
      | \\
      \{?template:Template,\\
      ?departure:DateTime,\\
      arrival:DateTime,\\
      ?airplaneType:AirplaneType,\\
      ?duringInterval:TimePeriod\}\\
      | \\
      \{?template:Template,\\
      ?departure:DateTime,\\
      ?arrival:DateTime,\\
      ?airplaneType:AirplaneType,\\
      duringInterval:TimePeriod\}
    }
\end{description}

\subsubsection{Operations}
Flight operations are
\paragraph{Adding a flight}
\begin{quote}
  \tye{ADD FLIGHT <Flight\_data>}
\end{quote}
\paragraph{Changing a template}
\begin{quote}
  \tye{CHANGE FLIGHT <Flight> TO <Flight>}
\end{quote}
\paragraph{Removing a template}
\begin{quote}
  \tye{REMOVE FLIGHT <Flight>}
\end{quote}


\subsection{Period}
\subsubsection{Types}
Period types are
\begin{description}
  \item[\ty{ContainedPeriod}] \tye{= \{?from:Data, ?to:Date, ?day:Date\}}
  \item[\ty{Period}] \tye{= \{?contained:ContainedPeriod,\\
      ?from:Date,\\
      ?to:Data,\\
      ?day:String,\\
      ?startTime:Time\}}
  \item[\ty{Period\_data}] \tye{= \{?from:Date,\\
      ?to:Data,\\
      ?weekday:String,\\
      startTime:Time\}}
\end{description}

\subsubsection{Operations}
Period operations are
\paragraph{Adding a period to a template}
\begin{quote}
  \tye{CHANGE TEMPLATE <Template> ADD PERIOD <Period\_data>}\\
  \tye{CHANGE TEMPLATE <Template> ADD PERIODS <[Period\_data]>}
\end{quote}
\paragraph{Changing a period of a template}
\begin{quote}
  \tye{CHANGE TEMPLATE <Template> CHANGE PERIOD <Period> TO <Period\_data>}\\
  \tye{CHANGE TEMPLATE <Template> CHANGE PERIODS [<Period>] TO <Period\_ data>}
\end{quote}
\paragraph{Removing a period from a template}
\begin{quote}
  \tye{CHANGE TEMPLATE <Template> REMOVE PERIOD <Period>}\\
  \tye{CHANGE TEMPLATE <Template> REMOVE PERIODS [<Period>]}
\end{quote}


\subsection{Seat instances}
\subsubsection{Types}
Seat instance types are
\begin{description}
  \item[\ty{SeatInstance}] \tye{= \{number:Int,\\
      ?amt:Int\} \\
      | \\
      \{type: String\}}
  \item[\ty{SeatInstance\_data}]  \tye{= \{number:Int,\\
      ?amt:Int,\\
      price:Price\}\\
      | \\
      \{type:String,\\
      price:Price\}}
\end{description}

\subsubsection{Operations}
Seat instance operations are
\paragraph{Changing seat instances of a template}
\begin{quote}
  \tye{CHANGE TEMPLATE <Template> CHANGE SEAT INSTANCES [<Seat>] TO <Seat\_data>}\\
  \tye{CHANGE TEMPLATE <Template> CHANGE SEAT INSTANCE <Seat> TO <Seat\_data>}\\
\end{quote}
\paragraph{Changing all seat instances of a template}
\begin{quote}
  \tye{CHANGE TEMPLATE <Template> CHANGE SEAT INSTANCES TO [<Seat\_data>]}\\
\end{quote}
\paragraph{Changing seat instances of a flight}
\begin{quote}
  \tye{CHANGE FLIGHT <Flight> CHANGE SEAT INSTANCES [<Seat>] TO <Seat\_data>}\\
  \tye{CHANGE FLIGHT <Flight> CHANGE SEAT INSTANCE <Seat> TO <Seat\_data>}\\
\end{quote}
\paragraph{Changing all seat instances of a flight}
\begin{quote}
  \tye{CHANGE FLIGHT <Flight> CHANGE SEAT INSTANCES TO [<Seat\_data>]}\\
\end{quote}

