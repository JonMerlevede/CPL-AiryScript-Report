\section{Types and operations}
\label{sec:operations}

\subsection{Basic types}
\begin{description}
  \item[\ty{Time}] \tye{= \{?h:Int, ?m:Int, ?s:Int\}}

    Specifies either a time period or point in time where \tye{h} stands for the
    number of hours, \tye{m} for the number of minutes and \tye{s} for the
    number of seconds. The default value of all properties is zero.
  \item[\ty{Date}] \tye{= \{d:Int, m:Int, y:Int\}}

    Specifies a date. \tye{d} stands for the day, \tye{m} for the month and
    \tye{y} for the year.
  \item[\ty{DateTime}] \tye{= \{date:Date, ?time:Time\}}

    Specifies a specific point in time. The default value of the time is at
    midnight.
  \item[\ty{Price}] \tye{= \{?dollar:Int, ?cent:Int\} | \{?euro:Int, ?cent:Int\}}

    Specifies an amount of money in either dollar or euros. The default value of
    all properties is zero.
  \item[\ty{TimePeriod}] \tye{= \{from:DateTime, to:DateTime\}}

    Specifies the time period in between the points in time defined by
    \tye{from} and \tye{to}.
\end{description}

\subsection{City}
\subsubsection{Types}
City types are
\begin{description}
  \item[\ty{City}] \tye{= \{?name: String\}}
  \item[\ty{City\_data}] \tye{= \{name: String\}}
\end{description}
where \tye{name} specifies the name of the city. 

\subsubsection{Operations}
City operations are
\paragraph{Adding a city}
\begin{operation}
  \lstinline{ADD CITY <data:City_data>}
  \label{op:add_city}
\end{operation}
Adds the city specified by \tye{data}.

This operation fails if there already exists a city with the given
name \tye{data→name}.

\begin{texa}[add the city of New York.]
  \lstinline|ADD CITY {name: "New York"}|
\end{texa}

\paragraph{Changing a city}
\begin{operation}
    \lstinline|CHANGE CITY <selector:City> TO <change:City>|
    \label{op:change_city}
\end{operation}
Changes the city or cities specified by \tye{citySelector} to have the values
specified by \tye{cityChange}.

This operation fails if it would introduce multiple cities with the name
\tye{change->name}.

Airport(s) that rely on the changed city or cities selected by \tye{selector}
update as well.

\begin{texa}[change the name of the city of New York to York.]
  \lstinline|CHANGE CITY { name: "New York" } TO { name: "York" }|
\end{texa}

\paragraph{Removing a city}
\begin{operation}
  \lstinline{REMOVE CITY <selector:City>}
  \label{op:remove_city}
\end{operation}
Removes the city or cities specified by \tye{selector}. Removing a city also
removes the airports in that city.

This operation fails if there is a template that flies to or from any of the
airports situated in any of the cities selected by \tye{selector}.

\begin{texa}[remove the city York.]
  \lstinline|REMOVE CITY {name: "York"}|
\end{texa}


\subsection{Airport}
\subsubsection{Types}
Airport types are
\begin{description}
  \item[\ty{Airport}] \tye{= \{?short:String,\\
      ?name: String,\\
      ?city:City\}}
  \item[\ty{Airport\_data}] \tye{= \{short:String,\\
      name: String,\\
      city:City\}}
\end{description}
where
\begin{itemize}
  \item \tye{short} stands for the short code airlines use internally to
    represent airports.
  \item \tye{name} stands for the full-length name of the airport.
  \item \tye{city} stands for the city the airport is situated in.
\end{itemize}
\subsubsection{Operations}
Airport operations are
\paragraph{Adding an airport}
\begin{operation}
  \lstinline{ADD AIRPORT <data:Airport_data>}
  \label{op:add_airport}
\end{operation}
Adds the airport specified by \tye{data}.

This operator fails if the \tye{data→city} property does not select
precisely one city or there already exists an airport with the given
\tye{data→short} property in the database.

\begin{texa}[add JFK Airport to the city of New York.]
  \lstinline|ADD AIRPORT {short: JFK, name:"JFK Airport", city: {name: "York"}}|
\end{texa}

\paragraph{Changing an airport}
\begin{operation}
  \lstinline{CHANGE AIRPORT <selector:Airport> TO <change:Airport>}
  \label{op:change_airport}
\end{operation}
Changes the airport or airports specified by \tye{selector} to have the values
specified by \tye{change}.

This operation fails if it would introduce multiple cities with the short
\tye{change→short}.

Template(s) that fly to or from the airport(s) selected by \tye{selector} update
as well.

\begin{texa}[change JFK Airport to be in the city of York.]
  \lstinline|CHANGE AIRPORT {short: JFK} TO {city: {name: "York"}}|
\end{texa}


\paragraph{Removing an airport}
\begin{operation}
  \lstinline{REMOVE AIRPORT <selector:Airport>}
  \label{op:remove_airport}
\end{operation}
Removes the airport or airports specified by \tye{selector}. Also removes all
distances and flight times that refer to (any of) the deleted airport(s).

This operation fails if there is a template that flies to or from any of the
airports selected by \tye{selector}.

\begin{texa}[remove JFK Airport.]
  \lstinline|REMOVE AIRPORT {short: JFK}|
\end{texa}


\subsection{Distance}
\subsubsection{Types}
Distance types are
\begin{description}
  \item[\ty{Distance}] \tye{= \{?from:Airport,\\
      ?to:Airport,\\
      ?dist:Int\}}
  \item[\ty{Distance\_data}] \tye{= \{from:Airport,\\
      to:Airport,\\
      dist:Int\}}
\end{description}
where \tye{dist} is the distance of the flight trajectory from \tye{from} to
\tye{to}.
\subsubsection{Operations}
Distance operations are as follows.

\paragraph{Adding a distance}
\begin{operation}
  \lstinline{ADD DISTANCE <data:Distance_data>}
  \label{op:add_distance}
\end{operation}
Adds the distance specified by \tye{data}.

This operation fails if there already exists a distance from \tye{data→from} to
\tye{data→to}.

\paragraph{Changing a distance}
\begin{operation}
  \lstinline{CHANGE DISTANCE <selector:Distance> TO <change:Distance>}
  \label{op:change_distance}
\end{operation}
Changes the distances specified by \tye{selector} to have the values specified
by \tye{change}.

This operation fails if it would introduce multiple distances with the same
values for \tye{from} and \tye{to}.

\paragraph{Removing a distance}
\begin{operation}
  \lstinline{REMOVE DISTANCE <selector:Distance>}
  \label{op:remove_distance}
\end{operation}
Removes the distance specified by \tye{selector}.

This operation fails if executing it would delete a distance from airport $a$ to
airport $b$ for which there exists a template that flies from $a$ to $b$.

\subsection{Flight time}
\subsubsection{Types}
Flight time types are
\begin{description}
  \item[\ty{FlightTime}] \tye{= \{?from:Airport,\\
      ?to:Airport,\\
      ?type:AirplaneType,\\
      ?time:Time\}}
  \item[\ty{FlightTime\_data}] \tye{= \{from:Airport,\\
      to:Airport,\\
      type:AirplaneType,\\
      time:Time\}}
\end{description}
where \tye{time} is the time it takes an airplane of type \tye{type}
to travel from \tye{from} to \tye{to}.

\subsubsection{Operations}
Flight time operations are as follows.

\paragraph{Adding a flight time}
\begin{operation}
  \lstinline{ADD FLIGHT TIME <data:FlightTime_data>}
  \label{op:add_flighttime}
\end{operation}
Adds the flight time specified by \tye{data}.

This operation fails if either \tye{data→from} or \tye{data→to} selects multiple
airports, if \tye{data→type} matches multiple airplane types or if a flight time
for the combination of \tye{data→from}, \tye{data→to} and \tye{data→type} already
exists in the database.

\paragraph{Changing a flight time}
\begin{operation}
  \begin{lstlisting}
CHANGE FLIGHT TIME <selector:FlightTime>
TO <change:FlightTime>
  \end{lstlisting}
  \label{op:change_flighttime}
\end{operation}
Changes the flight time or flight times specified by \tye{selector} to have the
values specified by \tye{change}.

This operation fails if it would introduce multiple flight times  with the same
values for \tye{from}, \tye{to} and \tye{type}.

\paragraph{Removing a flight time}
\begin{operation}
  \lstinline{REMOVE FLIGHT TIME <selector:FlightTime>}
  \label{op:remove_flighttime}
\end{operation}
Removes the flight time or flight times specified by \tye{selector}.

This operation fails if it would delete a time it takes an airplane of type $t$
to travel from airport $a$ to airport $b$ for which there exists a template that
flies from $a$ to $b$ with a template of type $t$.

\subsection{Airplane type}
\subsubsection{Types}
Airplane type types are
\begin{description}
  \item[\ty{AirplaneType}] \tye{= \{?name:String\}}
  \item[\ty{AirplaneType\_data}] \tye{= \{name:String\}}
\end{description}
where \tye{name} is the name of the airplane.

There are also seats in airplanes. Seat types are as follows.
\begin{description}
  \item[\ty{Seat}] \tye{= \{number:Int, ?amt:Int\}\\
    | \{?type:SeatType\}}
    
    The default value for \ty{amt} is one.

  \item[\ty{Seat\_change}] \tye{= \{number:Int, ?amt:Int, typeChange:SeatType\}\\
    | \{?typeSelect:SeatType, typeChange:SeatType\}}

    The default value for \ty{amt} is one.

  \item[\ty{Seat\_data}] \tye{= \{number:Int, ?amt:Int, type:SeatType\}}

    The default value for \ty{amt} is one.
\end{description}

\subsubsection{Operations}
Seat numbers have to be unique within an airplane type, but do not have to be
continuous (i.e. there can be a seat with number $n$ even if not all the seats
from 1 to $n-1$ exist). Overlapping seat numbers will cause any of the
operations below to fail.

Changing the seating arrangement of an airplane type by adding, changing or
removing seats has an influence on all templates and/or flights with that
airplane type, because it has an influence on seat instances and seat instance
templates.
\begin{itemize}
  \item When removing seat $s$ from an airplane type, AiryScript deletes all
    (seat and seat template) instances of $s$, including the pricing
    information they contain.
  \item When adding a seat $s$ to an airplane type, AiryScript creates new
    instances of $s$, for which no pricing information is available.
\end{itemize}
The documentation of the methods below specifies how AiryScript handles this.
Generally, AiryScript tries to use seat types to predict sensible prices.

\paragraph{Adding an airplane type}
\begin{operation}
  \begin{lstlisting}
ADD AIRPLANE TYPE <data:AirplaneType_data>
WITH SEATS <seats:[Seat_data]>
  \end{lstlisting}
  \label{op:add_airplane_type}
\end{operation}
Adds the airplane type specified by \tye{data} with the seat arrangement
specified in \tye{seats}. For each element in \ty{Seat\_data}, AiryScript adds
the amount of seats specified by \tye{amt} of the type specified by \tye{type},
starting numbering at the seat number specified by \tye{number}.  The default
value for \tye{amt} is one, the default value of \tye{number} is equal to the
highest seat number so far plus one. AiryScript processes the seats in the order
they appear in the \tye{seats} list.

This operation fails if there already exists an airplane type with name
\tye{data→name}, or if \tye{seats→type} does not select precisely one seat type.


\begin{texa}[\label{ex:add_airplane}
Adds an airplane type with the name ‘Boeing 007’.
The airplane type has 100 seats: 10 first class seats, 20 business seats
and 70 economy seats. We assume these seat types already exist. The first class
seats are numbered from 1 to 10 (inclusive), the business seats are numbered 51
to 70 (inclusive) and the economy class seats are numbered from 11 to 50
(inclusive) and 71 to 100 (inclusive).
  ]
  {
\begin{lstlisting}
ADD AIRPLANE TYPE {
    name: "Boeing 007",
} WITH SEATS [
    {type: first, amt: 10},
    {type: business, number:51, amt: 20},
    {type: economy, amt: 40},
    {type: economy, amt: 30}
]
\end{lstlisting}
  }
\end{texa}

\paragraph{Adding seats to an airplane type}
\begin{operation}
  \begin{lstlisting}
ADD SEATS <datas:[Seat_data]>
TO AIRPLANE TYPE <selector:AirplaneType>
  \end{lstlisting}
  \label{op:add_seats}
\end{operation}
Adds all of the seats specified by \tye{datas} to the airplane or airplanes
selected by \tye{selector}.

When adding seats to airplane types, AiryScript adds seat template instances and
seat instances for all templates and flights that use that airplane type. 
When adding an instance $s$ with seat type $t$ linked to a template or flight
$L$, AiryScript uses the following rules.
\begin{itemize}
  \item If we add the first seat of the airplane type, equivalently if no other
    instances are linked to $L$, AiryScript sets the price of $s$ to zero.
  \item Otherwise, if there were already seats of type $t$ in the airplane type,
    or equivalently if there were already instances of type $t$ linked to $L$,
    AiryScript sets the price of $s$ to the price of the instance of type $t$
    that is linked to $L$ that has the lowest seat number.
  \item Otherwise, if $s$ is the first seat of type $t$ to be added, it gets the
    lowest price of any of the seat instances linked to $L$.
\end{itemize}

\paragraph{Changing an airplane type}
\begin{operation}
  \begin{lstlisting}
CHANGE AIRPLANE TYPE <selector:AirplaneType>
TO <change:AirplaneType>
  \end{lstlisting}
\end{operation}
Changes the airplane type or airplane types specified by \tye{selector} to have
the values specified by \tye{change}.
%\begin{texa}[Changes the name of the ‘Boeing 007’ airplane to ‘Boeing 666’.]
%  {
%    \begin{lstlisting}
%CHANGE AIRPLANE TYPE {
%  name: "Boeing 007"
%} TO {
%  name: "Boeing 666"
%}
%      \end{lstlisting}
%  }
%\end{texa}

%\paragraph{Changing seats of an airplane type}
%\begin{operation}
%  \begin{lstlisting}
%CHANGE SEATS <seatSelectors:[Seat]>
%OF AIRPLANE TYPE <airplaneSelector:AirplaneType>
%TO <change:Seat>
%  \end{lstlisting}
%  \label{op:change_seats}
%\end{operation}
%Changes the seat or seats selected that match any of the seat selectors in
%\tye{seatSelectors} and the airplane selector \tye{airplaneSelector} to have the
%values specified in \tye{change}.
%\begin{itemize}
%  \item If any of the seat selectors in \tye{seatSelector} specifies a seat
%    number $n$, that seat selector selects the (single) seat with number $n$,
%    unless it specifies a seat type that does not match the type of seat $n$.
%  \item
%If \tye{change} specifies a seat number $n$ and $m$ seats are selected,
%the selected seat with the lowest number gets seat number $n$, the selected
%seat with the second lowest number gets seat number $n+1$ and so on. If any
%of the seat numbers in the range $[n,n+m-1]$ are already taken, the
%operation is unsuccessful and does not change any seats.
%\end{itemize}
%Note that the change seats operation is currently capable of merging seat types,
%but not of changing the number of seats in seat types. This effect can be
%achieved by adding and removing seats or by redefining the entire seat
%arrangement of the airplane type.
%
%\vspace{\baselineskip}

\paragraph{Changing the seats of an airplane type}
\begin{operation}
  \label{op:change_seats}
  \begin{lstlisting}
CHANGE SEATS OF AIRPLANE TYPE <elector:AirplaneType>
TO <data:[Seat_change]>
  \end{lstlisting}
  \label{op:change_seats_of}
\end{operation}
Changes the seats of the airplane type or airplane types selected by
\tye{selector} according to the seat changes in \tye{data}.

\begin{itemize}
  \item A seat change that specifies \ty{number} defines the type of the seats
    in the range $[\ty{number},\ty{number}+\ty{amt}-1]$ to type \ty{typeChange}.
    The default value of \ty{amt} is one.

  \item A seat change that specifies \tye{typeSelect} changes the type of all
    seats of type \ty{typeSelect} to type \ty{typeChange}.

  \item A seat change that does not specify \ty{number} nor \ty{typeSelect}
    changes the type of all seats to \tye{typeChange}.

  \item Seat changes are processed in the order they are defined in \tye{data}.
\end{itemize}

This operation has a different effect than first removing
seats of the airplane type that are changed by any of the seat changes in
\tye{data} and then adding them again with the values from the last matching
seat change in \tye{data}. This is because removing and adding seats removes
the pricing information of all flights and templates of all airplane types
matching \tye{selector}.
%This operation converts the pricing
%information from before the change of airplane type according to the rules
%specified in the documentation of \opref{op:change_seats}.

AiryScript solves the problem of lost seat instance data (see above) by setting
the prices of all seats with seat type $t$ in the new airplane type to the same
price: the price of the seat with type $t$ in the airplane type before the
change that has the lowest seat number. If the old airplane has no seats of type
$t$, it sets the price to the price of least expensive seat on the old airplane.
Note that this means that even instances of seats that are not selected by any
of the selectors in \tye{seatSelectors} can be influenced.

This operation fails if \tye{data→type} does not select precisely one seat type.

\begin{texa}[Changes the Boeing 007 from example \ref{ex:add_airplane} to have three
  more business seats: seats number one and three (which used to be first class
seats) and seat number 100 (which used to be an economy seat)]
  {
  \begin{lstlisting}
CHANGE SEATS OF AIRPLANE TYPE {
  name: "Boeing 007"
} TO [
  {number: 1, typeChange: business},
  {number: 3, typeChange: business},
  {number: 100, typeChange: business}
]
  \end{lstlisting}
  }
\end{texa}

\begin{texa}[Changes the Boeing 007 from Example \ref{ex:add_airplane}, turning
    all first class seats into business seats. Assuming flight template ABC300
  as in Example \ref{ex:add_template}, after the change all business seats on
ABC300 flights (with default pricing) cost 100 dollar and all economy seats on
ABC300 flights (with default pricing) cost 5 dollar.]
  {
  \begin{lstlisting}
CHANGE OF AIRPLANE TYPE {
  name: "Boeing 007"
} TO {
  typeSelect: first, typeChange:business
}
  \end{lstlisting}
  }
\end{texa}
\paragraph{Removing an airplane type}
\begin{operation}
  \lstinline|REMOVE AIRPLANE TYPE <selector:AirplaneType>|
\end{operation}
Removes the airplane type(s) specified by \tye{selector}. 

This operation fails if there is a template or flight that uses any of the
airplane types selected by \tye{selector}.

\paragraph{Removing seats from an airplane type}
\begin{operation}
  \begin{lstlisting}
REMOVE SEAT <seatSelector:Seat>
FROM AIRPLANE TYPE <airplaneSelector:AirplaneType>
  \end{lstlisting}
\end{operation}
Removes the seat(s) selected by \tye{seatSelector} from the airplane type(s)
selected by \tye{airplaneSelector}.

\tye{amt} specifies the maximum number of seats to remove. If \tye{number} is
not specified but \tye{amt} is, AiryScript deletes the seats with the highest
seat numbers that match \tye{seatSelector}.

\begin{texa}[Removes ten business seats from the Boeing 007 as it is defined in
    ‘Adding an airplane type’. After this operation, the Boeing still contains
    first class seats numbered 1 to 10 and economy seats numbered 11 to 50 and
    71 to 100, but only the business seats that are numbered 51 to 60 remain;
  the seats numbered 61 to 70 are gone.]
  {
  \begin{lstlisting}
REMOVE SEATS [
  {amt: 10, type: economy}
] FROM AIRPLANE TYPE {
  name: "Boeing 007"
}
\end{lstlisting}
  }
\end{texa}

\subsection{Seat types}

\subsubsection{Types}
The only seat type type is the following.
\begin{description}
  \item[\ty{SeatType}] \tye{= String}
\end{description}
Seat types correspond simply to strings, which is the name of the seat type
(e.g. business, first class, …). Still, users need to add seat types before
they can be used.

Note that in the future we can still extend \tye{SeatType} by making it a union
type, without breaking backwards compatibility.

\paragraph{Adding a seat type}
\begin{operation}
  \lstinline{ADD SEAT TYPE <data:SeatType>}
\end{operation}
Adds the seat type specified by \tye{data}.

This operation fails if there already exists a seat type with the name
\tye{data}.

\begin{texa}[\label{ex:add_seat_type} Adds the seat types economy, business and
  first class.]
  \begin{lstlisting}
ADD SEAT TYPE {business}
ADD SEAT TYPE economy
ADD SEAT TYPE "first class"
  \end{lstlisting}
\end{texa}

\paragraph{Changing a seat type}
\begin{operation}
  \lstinline{CHANGE SEAT TYPE <selector:SeatType> TO <change:SeatType>}
\end{operation}
Changes the seat type selected by \tye{selector} to the seat type specified by
\tye{change}.

All seats using the seat type selected by \tye{selector} update as well.

This operation fails if it would introduce multiple seat types with the same
values.

\begin{texa}[Renames the seat type first class from Example
  \ref{ex:add_seat_type} to first for convenience.]
  \begin{lstlisting}
CHANGE SEAT TYPE "first class" TO first
  \end{lstlisting}
\end{texa}

\paragraph{Removing a seat type}
\begin{operation}
  \lstinline{REMOVE SEAT TYPE <selector:String>}
\end{operation}
Removes the seat type selected by \tye{selector} from the database.

This operation fails if there is a seat that has a seat type selected by
\tye{selector} as its type.

\subsection{Airline}
\subsubsection{Types}
Airline types are
\begin{description}
  \item[\ty{Airline}] \tye{= \{?short:String, ?name:String\}}
  \item[\ty{Airline\_data}] \tye{= \{short:String, name:String\}}
\end{description}
where \tye{short} is the short code (three characters) that identifies an
airline, and \tye{name} is the full-length name of the airline.

\subsubsection{Operations}
Flight time operations areas follows.

\paragraph{Adding an airline}
\begin{operation}
  \lstinline{ADD AIRLINE <data:Airline_data>}
\end{operation}
Adds the airline specified by \tye{data}.

This operation fails if there already exists an airline with the short name
\tye{data→short}.

\paragraph{Changing an airline}
\begin{operation}
  \lstinline{CHANGE AIRLINE <selector:Airline> TO <change:Airline>}
\end{operation}
Changes the airline(s) selected by \tye{selector} to have the values specified
by \tye{change}.

Template(s) that rely on the changed airline(s) selected by \tye{selector}
update as well.

\paragraph{Removing an airline}
\begin{operation}
  \lstinline{REMOVE AIRLINE <selector:Airline>}
\end{operation}
Removes the airline(s) selected by \tye{selector}.

This operation fails if there is a template bound to any of the airlines
selected by \tye{selector}.


\subsection{Templates}
\subsubsection{Types}
Template types are
\begin{description}
  \item[\ty{Template}] \tye{= \{?airline:String,\\
      ?fln:String,\\
      ?from:Airport,\\
      ?to:Airport,\\
      ?type:AirplaneType\}}
  \item[\ty{Template\_change}] \tye{= \{?fln:String,\\
      ?from:Airport,\\
      ?to:Airport,\\
      ?type:AirplaneType\}}
  \item[\ty{Template\_data}] \tye{= \{fln:String,\\
      from:Airport,\\
      to:Airport,\\
      type:AirplaneType\}}
\end{description}
where \tye{fln} is the flight number of the template, \tye{airline} is the
airline that operates the template, \tye{from} is the airport the template flies
from, \tye{to} is the airport the template flies to and \tye{type} specifies the
type of aircraft the template uses.

Templates also define seat template instances and are associated with periods.
Seat template instance types are as follows.
\begin{description}
  \item[\ty{SeatTemplateInstanceChanger}] \tye{= \{number:Int,
      ?amt:Int,
      price:Price\} \\
    | \{?type:SeatType, price:Price\}}

    \tye{amt} has a default value of one.
\end{description}

Period types are as follows.
\begin{description}
  \item[\ty{Period\_data}] \tye{= \{?from:Date,\\
    ?to:Date,\\
    ?weekday:String,\\
    startTime:Time\}}

    
  \item[\ty{ContainedPeriod}] \tye{= \{?from:Date, ?to:Date, ?day:Date\}}

    A helper type to select periods. The default values of all properties is
    ‘don’t care’.
    A contained period $cp$ matches all periods $p$ that have
    \begin{itemize}
      \item A $p$\tye{→from} $\leq$ $cp$\tye{→from} in the case of a specified
        $cp$\tye{→from},
      \item a $p$\tye{→to} $\geq$ $cp$\tye{→to} in the case of a specified $cp$\tye{→to},
      \item a $p$\tye{→from} $\geq$ $cp$\tye{→day} and a $p$\tye{→to} $\leq$
        $cp$\tye{→day} in the case of a specified $cp$\tye{→day},
      \item the intersection of the above in case of multiple specified
        properties.
    \end{itemize}

  \item[\ty{Period}] \tye{= \{?contained:ContainedPeriod,\\
    ?from:Date,\\
    ?to:Date,\\
    ?weekday:String,\\
    ?startTime:Time\}}

    %Period \tye{contained} can only be used for selecting existing period
    %instances.
    A \tye{Period} $p$ selects periods that exactly match the $p$\ty{→from},
    $p$\ty{→to}, $p$\ty{→weekday} and $p$\ty{→startType} that are specified
    (like usual), and, if specified, match \ty{contained}.

    \begin{texa}[Selects all periods that occur \emph{only} on
      Friday. This matches the period of Example \ref{ex:period}, but not the
    period of Example \ref{ex:period2}.][skip]
      \ty{\{weekday: friday\}}
    \end{texa}
    \begin{texa}[Selects all periods that are valid on the day 5/7/2013. This
      matches both the periods from Example \ref{ex:period} and Example
    \ref{ex:period2}. Note that 5/7/2013 is a Friday.][skip]
      \ty{\{contained: \{day: \{d: 5, m: 7, y: 2013\}\}\}}
    \end{texa}
\end{description}


\subsubsection{Operations}
Template operations are as follows.
\paragraph{Adding a template}
\begin{operation}
  \label{op:add_template}
  \begin{lstlisting}
ADD TEMPLATE <data:Template_data>
WITH SEAT INSTANCES <seatDatas:[SeatTemplateInstanceChanger]>
AND WITH PERIODS <periodDatas:[Period_data]>
  \end{lstlisting}
\end{operation}
Adds the template specified by \tye{data}, \tye{seatDatas} and
\tye{periodDatas}.

Seat instances are processed in the order in which they appear in
\tye{seatDatas}. This can be important, because different seat data elements in
\tye{seatDatas} can override each other (see Example \ref{ex:add_template}).

The order in which the periods appear in \tye{periodDatas} is not relevant. If
the date \tye{periodDatas→from} is not specified, its default value is the
current time.


The operation fails if any of the following is true.
\begin{itemize}
  \item \tye{data→fln} is not correctly structured.
  \item There already exists a template with flight number \tye{data→fln}.
  \item The airline specified by \tye{data→fln} does not exist.
  \item \tye{data→type} does not select precisely one airplane type.
  \item \tye{data→from} or \tye{data→to} does not select precisely one airport.
  \item Some seats in the airplane type \tye{data→type} are not assigned a price
    by \tye{seatDatas}.
  \item \tye{seatDatas} specifies a seat number that does not exist in the
    airplane type \tye{data→type}.
\end{itemize}

The periods in \ty{periodDatas} specify how templates recur. If $p$ is a period
in \ty{periodDatas}. Then $p$\tye{→from} specifies the date starting from which
a period is valid, $p$\tye{→to} specifies the date until when a period is valid,
$p$\tye{→weekday} specifies a weekday and $p$\tye{→startTime} specifies the time
of day when airplanes depart.
    \begin{itemize}
      \item Weekday can take values \ty{monday}, \ty{tuesday}, \ty{thursday},
        \ty{wednesday}, \ty{friday}, \ty{saturday} and \ty{sunday}. Any other
        values cause a parsing error.
      \item If \ty{weekday} is not specified, the period is valid on all
        weekdays, i.e., specifies seven flights a week, with there being a
        flight every day of the week at \tye{startTime}.
        
        If \ty{weekday} is specified, the period is only valid on the specified
        weekday, i.e., specifies one flight at \tye{startTime} a week.
      \item The default value of \tye{from} is the current time.
      \item The default value of \tye{to} is infinity, i.e. the period never
        ends.
      \begin{texa}[\label{ex:period}A \ty{Period\_data} instance that specifies
        flights at 12:00 every friday from now until the end of times.][skip]
        \ty{\{weekday: friday, startTime: {h: 12}\}}
      \end{texa}
      \begin{texa}[\label{ex:period2}A \ty{Period\_data} instance that specifies flights at
        23:44 every day of the year 2013.][skip]
        \ty{\{from: {d:1, m:1, y:2013},\\
        to: {d: 31, m: 12, y:2013},\\
        startTime: {h: 23, m:44}\}}
      \end{texa}
    \end{itemize}
If there is a template $T$ with a period $p$, this means that there is a
flight instance of $T$ at the hour specified by $p$ on all the days
specified by $p$. Note that $p$ can specify an infinite number of days by
leaving out the end time, and therefore an infinite number of flights.
AiryScript immediately allows users to change any of these flights, at any
time, right after creating $T$ with $p$.

There can be multiple identical periods bound to $T$. By adding an
identical period twice, there will be two flights of $T$ at 12:00 every
Friday from now until the end of times.
\begin{texa}[A template that is bound to the period specified by Example
    \ref{ex:period} twice, has two flights at 12:00 every friday from now
  until the end of times.]
\end{texa}
\vspace{\baselineskip}
A template that is not linked to any period is a non-recurring template.
These templates can still be useful, because users can also add flights
manually using \opref{op:add_flight}, but always need to bind them to a
template.

\begin{texa}[\label{ex:add_template}
    Adds template ABC300, which flies from Zürich airport to Zaventem
    using a Boeing 007 airplane. First class seats cost 200 dollar, business
    class seats cost 100 dollar and economy class seats cost 50 dollar, except
    from seat number 11, which costs only 5 dollar, and seat number 12, which
    costs 5000 dollar. ABC300 flies every tuesday at 12:55 for five years, from
    28/12/2015 to 28/12/2020.]
  \begin{lstlisting}
ADD TEMPLATE {
    fln: ABC300,
    from: {city: {name: Zürich}},
    to: {name: Zaventem}
    type: "Boeing 007"
} WITH SEAT INSTANCES [
    {type: first, price: {dollar: 200}},
    {type: business, price: {dollar: 100}},
    {type: economy, price: {dollar: 50}},
    {number: 11, amt: 2, price: {dollar: 5000}}
    {number: 11, price: {dollar: 5}}
] AND WITH PERIODS [{
    day: tuesday,
    departure: {h: 12, m:55},
    from: {d: 28, m:12, y:2015},
    to: {d: 28, m: 12, y:2020}
}]
  \end{lstlisting}
\end{texa}

\paragraph{Adding periods to a template}
\begin{operation}
  \label{op:add_periods}
  \begin{lstlisting}
ADD PERIODS <datas:[Period_data]>
TO TEMPLATE <selector:Template>
  \end{lstlisting}
\end{operation}
Adds the periods specified by \tye{datas} to the template selected by
\tye{selector}. This new period specifies new flights as described in the
documentation of \opref{op:add_template}.

This operation fails if \tye{selector} does not match precisely one template.

\begin{texa}[\label{ex:add_periods}Specifies a recurring period for the template
  ABC300 of Example \ref{ex:add_template}, so that there is an extra flight of
  ABC300 every Friday during the summer vacation of 2013.]
  \begin{lstlisting}
ADD PERIODS {
    weekday: friday,
    from: {d: 1, m: 7, y:2013},
    to: {d: 1, m: 8, y:2013}
} TO TEMPLATE { fln: ABC300 }
  \end{lstlisting}
\end{texa}


\paragraph{Changing a template}
\begin{operation}
  \label{op:change_template}
  \begin{lstlisting}
CHANGE TEMPLATE <selector:Template>
TO <change:Template>
  \end{lstlisting}
\end{operation}
Changes the templates specified by \tye{selector} to have the values specified
by \tye{change}.

If \tye{change} specifies a change of airplane type, this operation tries to
convert prices of seat instance templates in the same way as
\opref{op:change_seats}.

Flights that are bound to a changed template may change together with changes to
the template.
\begin{itemize}
  \item AiryScript never changes flights with a departure time that lies in the
    past (i.e. a departure time smaller than the current time).
  \item AiryScript never changes properties of flights that a user changed
    manually.
    \begin{itemize}
      \item If a user specified a price for a seat on a specific flight,
        changing the price of that seat in the underlying flight template has no
        effect on the price of the seat on that flight.

        If a user did not specify a specific price for a seat on a specific
        flight, changing the price of that seat in the underling flight template
        causes the price of that seat to change together with the price of that
        seat in the template.
      \item If a user specified a specific departure time for a flight, that
        departure time no longer changes together with the time specified in
        template.
      \item If a user specified a specific airplane type for a flight, that
        airplane type no longer changes together with the airplane
        type of the template.

        Specifying a specific airplane type requires translating seat prices, as
        is explained in the section on airplane types and \opref{op:add_seats}.
        After specifying an airplane type, the translated seat prices also no
        longer change together with the template.
      \item If a user specified a specific arrival time, that arrival time no
        longer changes together with changes of the flight time associated with
        the template.
    \end{itemize}
    
    For example, if a user has specified that business seats on a
    flight cost 500 dollars, changing the template of that flight does not
    change the price of business seats. If the user used default values for the
    price of economy seats, those default values change together with changes to
    the template.
\end{itemize}
\begin{texa}\label{ex:change_template}
  \\
  Create a template where all seats cost 100 dollar. It flies from Brussels to
  JFK airport in New York, every day at 8 o’clock from 9/1/2013 until the end of
  times using a Boeing 007. We assume these airports and airplane types already
  exist.
  \begin{quote}\begin{lstlisting}
ADD TEMPLATE {
    fln: JON007,
    from: {city: {name:Brussels}}
    to: {short: JFK, city: {name:"New York"}}
    type: "Boeing 007"
} WITH SEAT INSTANCES {
    price: {dollar: 100}
} AND WITH PERIODS [
    from: {d: 9, m: 01, y: 2013}
    startTime: {h: 8}
]
  \end{lstlisting}\end{quote}
  We expect a lot of people will want to fly from Brussels to New York on
  9/1/2013, so we create an extra flight for this template at midnight that
  specifies a special price of 300 dollar for the business seats. Because this
  is the first flight in a series of many, the flight is made using a special
  party airplane.
  \begin{quote}\begin{lstlisting}
ADD FLIGHT {
    template: {fln: JON007},
    departure: {date: {d: 9, m:1, y: 2013}},
    type: Partyplane
} WITH SEAT INSTANCES {
    type: business, price: {dollar:300}
}
  \end{lstlisting}\end{quote}
  On this flight, all seats now cost 100 dollar, except from the business seats,
  which cost 300 dollar.
  We also expect ‘economy’ people are willing to pay more during the summer
  vacation, so we specify a higher economy seat price of 150 dollar for those
  flights.
  \begin{quote}\begin{lstlisting}
CHANGE SEAT INSTANCES OF FLIGHT {
    fln: JON007,
    duringInterval:
    {
        from: {d:1, m:7, y:2013},
        to: {d:1, m:8, y:2013}
    }
} TO {
    type: economy, price: {dollar: 150}
}
  \end{lstlisting}\end{quote}
  We now decide that 100 dollar for a seat is really never enough, so we change
  the default price of flights by changing the JON007 template.
  \begin{quote}\begin{lstlisting}
CHANGE SEAT INSTANCES OF TEMPLATE {
    fln: JON007
} TO {
    price: {dollar:200}
}
  \end{lstlisting}\end{quote}
  Now, on the first flight we created, all seats except the business seats still
  cost 100 dollar. The price of these seats did not change, because the first
  flight specified a special airplane type. The business seats on the first
  flight still cost 300 dollar.
  
  Seats on flights during the summer vacation all cost 200 dollar, except from
  the economy seats, which still cost 150 dollar. All seats on all other flights
  with the JON007 template cost 200 dollar. This are infinitely many flights and
  seats!

  We also change the airplane type to the newer Boeing 666. We suppose this new
  airplane type already exists.
  \begin{quote}\begin{lstlisting}
CHANGE TEMPLATE {fln: JON007}
TO {
    type: "Boeing 666"
}
  \end{lstlisting}\end{quote}
  All flights belonging to the JON007 template, also the flights in the summer
  vacation, now use the Boeing 666 airplane. That is, all except from the first
  flight, which still uses the party airplane.
\end{texa}
\vspace{\baselineskip}
This operation fails if it would introduce multiple templates with the same
flight number.

\paragraph{Changing seat template instances of a template}
\begin{operation}
  \label{op:change_seat_template_instance}
  \begin{lstlisting}
CHANGE SEAT INSTANCES OF TEMPLATE <selector:Template>
TO <changes:[SeatTemplateInstanceChanger]>
  \end{lstlisting}
\end{operation}
Changes the seat template instances of the template(s) selected by
\tye{selector} according to the seat template instance changes in \ty{changes}.
\begin{itemize}
  \item A seat template instance change $i\in\ty{changes}$ that specifies
    $i$\ty{→number} changes the price of the seat template instances with
    numbers in the range $[i\ty{→number},i\ty{→number}+i\ty{→amt}-1]$ to
    $i$\ty{→price}. The default value of $i$\ty{→amt} is one.

  \item A seat template instance change $i\in\ty{changes}$ that specifies
    $i$\tye{→type} changes the price of all seat template instances with type
    $i$\ty{→type} to $i$\ty{→price}.

  \item A seat template instance change $i\in\ty{changes}$ that does not specify
    $i$\ty{→number} nor $i$\ty{→type} changes the price of all seat template
    instances to $i$\tye{→price}.

  \item Seat template instance changes are processed in the order they are
    defined in \tye{data}. The effects of an instance changer can be overridden
    by an instance changer that is processed later on in the list.
\end{itemize}

\paragraph{Changing periods of a template}
\begin{operation}
  \label{op:change_periods}
  \begin{lstlisting}
CHANGE PERIOD <periodSelector:Period>
TEMPLATE <templateSelector:Template>
TO <change:Period_data>
  \end{lstlisting}
\end{operation}
Changes all periods selected by \ty{periodSelector} of the templates selected
by \ty{templateSelector} to have the values specified by \ty{change}.

Flights that rely on the changed periods change as well. Let $p$ be a changed
period before it is changed and $p’$ be that period after the change. Periods
affect flights as described in \opref{op:add_template}.

\begin{itemize}
  \item \opref{op:change_periods} leaves the flights bound to $p$ that $p’$
    specifies that $p$ also specifies unchanged. This includes flights with
    changed departure times.
  \item \opref{op:change_periods} initialises the flights bound to $p$ that $p’$
    specifies but $p$ does not specify are initialised with the defaults set in
    the template selected by \ty{templateSelector}.
  \item \opref{op:change_periods} removes the flights bound to $p$ that $p$
    specifies but $p’$ does not specify from the database. This includes all
    flights for which the departure time was changed. Removing a flight also
    removes all its seat instances.

    \opref{op:change_periods} makes an exception to this rule when
    $p$\ty{→weekday} and $p’$\ty{→weekday} are different but both specified. In
    that case, \opref{op:change_periods} shifts all existing flights for which
    the user did not change the departure time from the old day of the
    week to the new day of the week. \opref{op:change_periods} leaves the
    flights for which the user did manually change the departure time unchanged.
\end{itemize}

This operation fails if \tye{templateSelector} does not select exactly one
template.

\begin{texa}[\label{ex:change_periods}
    Changes the period(s) of the template ABC300 defined in Example
    \ref{ex:add_template} to be valid on Wednesdays instead of on tuesdays. If
    ABC300 contains only the period defined in Example
    \ref{ex:add_periods}, no flights are removed and no information is lost. If
    ABC300 also contains other periods, information on pricing is lost only if
    they were valid on all weekdays.]
  \begin{lstlisting}
CHANGE PERIOD { }
OF TEMPLATE {name: ABC300}
TO {weekday: wednesday}
  \end{lstlisting}
\end{texa}

\paragraph{Changing all periods of a template}
\begin{operation}
  \label{op:change_periods}
  \begin{lstlisting}
CHANGE PERIODS OF TEMPLATE <selector:Template>
TO <datas:[Period_data]>
  \end{lstlisting}
\end{operation}
This operation is syntactic sugar for first removing all periods from the
template specified by \tye{selector} using \opref{op:remove_periods} and then
adding all of the periods specified by \tye{datas} using
\opref{op:add_periods}.

\paragraph{Removing a template}
\begin{operation}
  \lstinline{REMOVE TEMPLATE <selector:Template>}
\end{operation}
Removes the templates selected by \tye{selector}. AiryScript also removes all
periods and future and past flights associated with the template along with all
pricing information on the template and all pricing information of the deleted
flights.

This is a dangerous operation. If an overview of past flights should be kept,
consider simply removing all periods from the template. The template then also
no longer specifies any recurring flights.

\begin{texa}[Remove all flights and flight templates from the system. The effect
  is similar to executing \ty{sudo rm -rf /} on UNIX systems.]
  \begin{lstlisting}
REMOVE TEMPLATE { }
  \end{lstlisting}
\end{texa}

\paragraph{Removing periods from a template}
\begin{operation}
  \label{op:remove_periods}
  \begin{lstlisting}
REMOVE PERIODS <periodSelectors:[Period]>
FROM TEMPLATE <templateSelector:Template>
  \end{lstlisting}
\end{operation}
Removes the periods selected by any of the period selectors in
\tye{periodSelectors} from all of the templates matched by
\ty{templateSelector}. Also deletes all flights that were created from any of
the deleted periods.


\begin{texa}[Removes all recurring flights that fly only on Mondays. The
  president of GlobAir. Inc. \emph{really} hates mondays.]
  \begin{lstlisting}
REMOVE PERIODS {
    day: monday
} FROM TEMPLATE { }
  \end{lstlisting}
\end{texa}
\begin{texa}[Removes all recurring flights that of the ABC300 template. This is
  an alternative to deleting the template that keeps information on past
flights.]
  \begin{lstlisting}
REMOVE PERIODS { }
FROM TEMPLATE {
    fln: ABC300
}
  \end{lstlisting}
\end{texa}


\subsection{Flights}
Flight types are as follows.
\subsubsection{Types}
\begin{description}
  \item[\tye{Flight}] \tye{= \{?template:Template,\\
      departure:DateTime,\\
      ?arrival:DateTime,\\
      ?type:AirplaneType,\\
      ?duringInterval:TimePeriod\}\\
      | \\
      \{?template:Template,\\
      ?departure:DateTime,\\
      arrival:DateTime,\\
      ?type:AirplaneType,\\
      ?duringInterval:TimePeriod\}\\
      | \\
      \{?template:Template,\\
      ?departure:DateTime,\\
      ?arrival:DateTime,\\
      ?type:AirplaneType,\\
      duringInterval:TimePeriod\}
    }
  \item[\ty{Flight\_change}] \tye{= \{?departure:DateTime,\\
      ?arrival:DateTime,\\
      ?type:AirplaneType\}}
  \item[\ty{Flight\_data}] \tye{= \{template: Template,\\
      departure: DateTime,\\
      ?arrival: DateTime,\\
      ?airplaneType: AirplaneType\}}
\end{description}
The \ty{Flight} type is long because either \ty{departure}, \ty{arrival} or
\ty{duringInterval} has to be specified. This is to prevent users from being
able to select an infinite number of flights. AiryScript is not capable of
processing infinite numbers of flights directly, but users can work with
infinite flight collections by specifying and manipulating template defaults and
template periods.

Users can also change properties related to seats of flights using the following
type.
\begin{description}
  \item[\ty{SeatTemplateInstanceChanger}] \tye{= \{number:Int,
      ?amt:Int,
      price:Price\} \\
    | \{?type:SeatType, price:Price\}}
\end{description}

\subsubsection{Operations}
Flight operations are as follows.


\paragraph{Adding a flight}
\begin{operation}
  \begin{lstlisting}
ADD FLIGHT <data:Flight_data>
  \end{lstlisting}
\end{operation}
Adds the flight specified by \ty{data}.
This operation is syntactic sugar for \opref{op:add_flight} with an empty list
of seat instance changers (\lstinline|ADD FLIGHT data WITH SEAT INSTANCES []|).





\paragraph{Adding a flight with seat instances}
\begin{operation}
  \label{op:add_flight}
  \begin{lstlisting}
ADD FLIGHT <flightData:Flight_data>
WITH SEAT INSTANCES <seatDatas:[SeatInstanceChanger]>
  \end{lstlisting}
\end{operation}
This operation adds a single flight with the data specified in \tye{flightData}
and seat instance changes specified by \tye{seatDatas}.

If the arrival time or airplane type are not specified, they are set to the
default values in the template. As long as the \tye{departure}

Using this operation, users can add single flights without specifying a new
period for the underlying template. Use \opref{op:add_periods} to add lots of
flights using only one operation.

Flights with a \tye{data→departure} that lies in the future that do not specify
prices for some seat instances, airplane type or arrival time use the default
values derived from the template \tye{data→template}. If the default values of
in \tye{data→template} changes, the (future) flights that use these default
values change as well. See the documentation of \opref{op:change_template} on
page \pageref{op:change_template} for all information and an extended example
that includes the creation of a flight using ADD FLIGHT.

This operation fails if \tye{flightData→template} does not match exactly one
template.

\paragraph{Changing a flight}
\begin{operation}
  \begin{lstlisting}
CHANGE FLIGHT <selector:Flight>
TO <change:Flight_change>
  \end{lstlisting}
\end{operation}
Changes the flight(s) selected by \ty{selector} according to the values
specified in \ty{change}.

Changing properties of a flight can have an influence on how the flight changes
together with changes to its underlying template. See the documentation of
\opref{op:change_template} for more information.

This operation fails if \tye{change→type} matches multiple airplane types.

\begin{texa}
  \label{ex:change_flight}
  We extend the code of Example \ref{ex:change_template} (page
  \pageref{ex:change_template}). ABC decides that it does not want to use the
  party airplane after all, and changes the type of airplane to Boeing 666.
  \begin{quote}\begin{lstlisting}
CHANGE FLIGHT {
  template: {fln: JON007},
  type: Partyplane
} TO {
  type: "Boeing 666"
}
  \end{lstlisting}\end{quote}
  Note that the prices of all seats this flight are still ‘decoupled’ from the
  default values defined in the JON007 template.
\end{texa}

\paragraph{Changing seat instances of a flight}
\begin{operation}
  \begin{lstlisting}
CHANGE SEAT INSTANCES OF <selector:Flight>
TO <changes:[SeatInstanceChanger]>
  \end{lstlisting}
\end{operation}
Changes the seat instances of the flight(s) selected by
\tye{selector} according to the seat template instance changes in \ty{changes}.
\begin{itemize}
  \item A seat instance change $i$ that specifies $i$\ty{→number} changes the
    price of seat instances with numbers in the range
    $[i\ty{→number},i\ty{→number}+i\ty{→amt}-1]$ to $i$\ty{→price}.  The
    default value of $i$\ty{→amt} is one.

  \item A seat instance change $i$ that specifies $i$\tye{→type} changes the
    price of all seat instances with type $i$\ty{→type} to type $i$\ty{→price}.

  \item A seat instance change $i$ that does not specify $i$\ty{→number}
    nor $i$\ty{→type} changes the price of all seat instances to
    $i$\tye{→price}.

  \item Seat template instance changes are processed in the order they are
    defined in \tye{data}. The effects of an instance changer can be overridden
    by an instance changer that is processed later on in the list.
\end{itemize}

\paragraph{Removing a template}
\begin{operation}
  \begin{lstlisting}
REMOVE FLIGHT <selector:Flight>
  \end{lstlisting}
\end{operation}
Removes the flight(s) selected by \ty{selector}. This also removes all seat
instances associated with those flights.

Users can also remove flights that were defined by a period and not by
\opref{op:add_flight}.

\begin{texa}
  Create a template that defines flights every day at 8h starting from 9/1/2013.
  \begin{lstlisting}
ADD TEMPLATE {
    fln: JON007,
    from: {city: {name:Brussels}}
    to: {short: JFK, city: {name:"New York"}}
    type: "Boeing 007"
} WITH SEAT INSTANCES {
    price: {dollar: 100}
} AND WITH PERIODS [
    from: {d: 9, m: 1, y: 2013}
    startTime: {h: 8}
]
  \end{lstlisting}
  Removes the flights from the above template during the first summer vacation.
  \begin{lstlisting}
REMOVE FLIGHT {
  template: {fln: JON007}
  duringInterval: {
      from: {d:1, m:7, y:2013},
      to: {d:31, m:8, y:2013}
  }
}
  \end{lstlisting}
\end{texa}


%\subsection{Period}
%\subsubsection{Types}
%Period types are
%\begin{description}
%  \item[\ty{ContainedPeriod}] \tye{= \{?from:Data, ?to:Date, ?day:Date\}}
%  \item[\ty{Period}] \tye{= \{?contained:ContainedPeriod,\\
%      ?from:Date,\\
%      ?to:Data,\\
%      ?day:String,\\
%      ?startTime:Time\}}
%  \item[\ty{Period\_data}] \tye{= \{?from:Date,\\
%      ?to:Data,\\
%      ?weekday:String,\\
%      startTime:Time\}}
%\end{description}
%
%\subsubsection{Operations}
%Period operations are
%\paragraph{Adding a period to a template}
%\begin{quote}
%  \tye{CHANGE TEMPLATE <Template> ADD PERIOD <Period\_data>}\\
%  \tye{CHANGE TEMPLATE <Template> ADD PERIODS <[Period\_data]>}
%\end{quote}
%\paragraph{Changing a period of a template}
%\begin{quote}
%  \tye{CHANGE TEMPLATE <Template> CHANGE PERIOD <Period> TO <Period\_data>}\\
%  \tye{CHANGE TEMPLATE <Template> CHANGE PERIODS [<Period>] TO <Period\_ data>}
%\end{quote}
%\paragraph{Removing a period from a template}
%\begin{quote}
%  \tye{CHANGE TEMPLATE <Template> REMOVE PERIOD <Period>}\\
%  \tye{CHANGE TEMPLATE <Template> REMOVE PERIODS [<Period>]}
%\end{quote}
%
%
%\subsection{Seat instances}
%\subsubsection{Types}
%Seat instance types are
%\begin{description}
%  \item[\ty{SeatInstance}] \tye{= \{number:Int,\\
%      ?amt:Int\} \\
%      | \\
%      \{type: String\}}
%  \item[\ty{SeatInstance\_data}]  \tye{= \{number:Int,\\
%      ?amt:Int,\\
%      price:Price\}\\
%      | \\
%      \{type:String,\\
%      price:Price\}}
%\end{description}
%
%\subsubsection{Operations}
%Seat instance operations are
%\paragraph{Changing seat instances of a template}
%\begin{quote}
%  \tye{CHANGE TEMPLATE <Template> CHANGE SEAT INSTANCES [<Seat>] TO <Seat\_data>}\\
%  \tye{CHANGE TEMPLATE <Template> CHANGE SEAT INSTANCE <Seat> TO <Seat\_data>}\\
%\end{quote}
%\paragraph{Changing all seat instances of a template}
%\begin{quote}
%  \tye{CHANGE TEMPLATE <Template> CHANGE SEAT INSTANCES TO [<Seat\_data>]}\\
%\end{quote}
%\paragraph{Changing seat instances of a flight}
%\begin{quote}
%  \tye{CHANGE FLIGHT <Flight> CHANGE SEAT INSTANCES [<Seat>] TO <Seat\_data>}\\
%  \tye{CHANGE FLIGHT <Flight> CHANGE SEAT INSTANCE <Seat> TO <Seat\_data>}\\
%\end{quote}
%\paragraph{Changing all seat instances of a flight}
%\begin{quote}
%  \tye{CHANGE FLIGHT <Flight> CHANGE SEAT INSTANCES TO [<Seat\_data>]}\\
%\end{quote}
%
